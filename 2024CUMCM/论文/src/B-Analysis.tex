% 设置页码计数器为 1 (也就是当前页面为第一页)
\setcounter{page}{1}

% ==================================================
% @brief    问题重述
% ==================================================

\mcmSection{问题重述}
\quad “板凳龙”是将少则几十条,多则几百条的板凳首尾相连,形成“盘龙”形状的地方民俗文化活动。盘龙时,龙头领头在前,龙身相继环绕,龙尾相随盘旋,外观上程圆盘状。在舞龙队能自由盘入盘出的情况下,盘龙所需的面积越小,盘龙的行进速度就越快,其观赏性就越好。


% 标题已经是问题重述了,这里不分小节
% 一两句简短的背景或没有,如果有的话,问题背景应用自己的语言精炼概况

% 问题重述的关键在于以自己的理解重新叙述题目所给问题(反映团队理解题意的深度)
\textbf {问题一:}
已知盘龙队各把手中心行进轨迹符合等距螺线方程,初始时刻龙头位于第16圈A点,龙头把手中心行进速度恒为$1m/s$,当舞龙队沿螺距为55cm的等距螺线顺时针盘入,建立板凳龙等距螺线行进轨迹的数学模型,求初始时刻到$300s$为止时,每秒龙头、龙身和龙尾各前把手中心及龙尾后把手中心的位置和速度,将结果保存到result.xlsx附件中,同时在上述模型的基础上,在文中以题目中给出的表1、表2格式,列出第0s、60s、120s、180s、240s、300s龙头前把手、龙头后面第1、51、101、151、201节龙身前把手和龙尾后把手的位置求解表格和速度求解表格。

\textbf {问题二:}
舞龙队按照问题1中所设定的螺旋路径进行盘绕,现需确定舞龙队盘绕至不再能够继续进行而不发生板凳间碰撞的终止时刻。请计算在此时刻舞龙队的位置与速度,并将这些数据记录到文件result2.xlsx中。此外,在论文中详细记录以下位置和速度信息:龙头前端把手的坐标和速度,以及龙头后第1、51、101、151、201节龙身前端把手和龙尾后端把手的坐标与速度。



\textbf {问题三:}在舞龙队从顺时针盘入转变为逆时针盘出的过程中,需要一定的空间来进行方向调整。假设调头空间为一个以螺线中心为圆心、直径为9米的圆形区域,需要计算出最小的螺距,这样龙头的最前端把手才能沿着指定的螺线顺利盘入,并触及调头空间的边缘。


\textbf {问题四:}螺线盘入阶段的螺距定为1.7米,而盘出螺线与盘入螺线相对于螺线中心对称。舞龙队需在问题3所定义的调头区域内完成转向,转向路径由两个半径不同的圆弧组成,形成S形的曲线,其中较大圆弧的半径是较小圆弧的两倍,且这两个圆弧都与盘入和盘出螺线相切。探究是否可以对圆弧进行调整,同时保持与其他部分的相切状态,从而缩短调头曲线的长度。

龙头前把手的移动速度持续为1米/秒。以调头启动时刻作为时间起点,记录从−100秒至100秒期间每秒钟舞龙队的具体位置与速度,详细记录时间点-100秒、-50秒、0秒、50秒、100秒龙头前把手以及龙头之后第1、51、101、151、201节龙身前把手和龙尾后把手的准确位置与速度。并将这些数据保存至result4.xlsx文件中。



\textbf {问题五:}
在遵循问题4所描述的路径前进时,舞龙队的龙头保持恒定的行进速度。需要确定的是龙头的最大行进速度,以确保舞龙队的所有把手速度均不超出2米/秒的限制。

% ==================================================
% @brief    问题分析
% ==================================================
\clearpage
\mcmSection{问题分析}

% 问题分析与问题重述最大的不同在于问题分析中给出求解方式。

\mcmSubsection{问题一的分析}
% 问题一的分析内容
舞龙队以螺距为55cm的等距螺线顺时针盘入,要求给出300秒内舞龙队的各处位置的坐标和速度。首先应该建立一个螺线方程用来描述龙头和各处板凳的位置
,螺线方程基于极坐标系,龙头速度已知为1m/s,即可确定螺线方程的参数。然后通过对时间积分,可求得其运动轨迹。而龙身和龙尾通过把手相连接,其运动轨迹需要考虑整个舞龙队的曲率的变化,因此需要通过数值模拟,以差分近似微分求数值解,进而得出整个龙队的位置和速度。
\mcmSubsection{问题二的分析}
% 问题二的分析内容
要求分析舞龙队在盘入过程中可能发生的碰撞,并确定何时达到碰撞的临界点。即相邻两节板凳不能太过接近。需要考虑板凳之间的相对运动和空间占用,以及螺线的几何限制。通过螺线轨迹方程,检测是否满足每节板凳间的最小安全距离,并设定其为临界条件。进而求得龙队无法继续盘入的最大时间,就能确定龙头和特定龙身节数的最终位置和速度。
\mcmSubsection{问题三的分析}
% 问题二的分析内容
确定最小螺距,使龙头能够盘入到直径为9m的掉头空间。需要计算调头空间所需的螺距,确保龙头前把手可以顺利进入调头区域。首先可以以螺线中心为圆心,建立半径为4.5m的圆形区域,作为掉头模型的空间区域,然后改变螺距,重新计算龙头盘入的路径,直到龙头抵达该空间边界。采用优化模型,可以求得既能够确保龙头进入掉头空间又不发生碰撞的最小螺距。


\mcmSubsection{问题四的分析}
% 问题一的分析内容
优化调头路径的几何形状,以减少调头所需的空间。设定掉头路线为由两段圆弧拼接而成的S形曲线,并建立曲线的方程,同时确保曲线与螺线相切。对曲线进行路径优化,或者调整圆弧半径的连接方式,计算曲线长度,以得到最小路径。

\mcmSubsection{问题五的分析}
在问题四的前提下,确定龙头的最大运行速度。约束限制舞龙队整体速度不超过2m/s,需要分析随着龙头速度增加时,队伍中靠近曲线内侧的部分速度会增大。通过数值仿真,逐步增加龙头速度,记录各节板凳的速度变化,当任意板凳的速度速度接近2m/s时,记录此时龙头的速度作为最大的行进速度。

% ==================================================
% @brief    模型假设
% ==================================================
\clearpage
\mcmSection{模型假设}
% 解题过程中用到的关键假设一定要写,非必要假设可有可无

\begin{enumerate}[label=\textbf{\arabic*.}]
	
	\item 恒定速度假定:假定在舞龙的表演过程中,其各个部分的移动速度保持不变,从而忽略了加速度产生的效应。
	
	\item 完美模型化假定:设想舞龙的各个部分之间的连接是完全灵活且不具备弹性特性,使得每一段龙身都能独立且自由地弯曲,不受其他部分的影响。
	
	\item 一致长度假定:假设舞龙每一节的长度均等,这一预设有助于简化和精确化数学计算与模型构建。
	
	\item 平面运动假定:认为舞龙的活动仅限于二维空间内,从而省略了其在垂直方向的运动以及重力的作用。
	
	\item 摩擦忽略假定:假设舞龙与地面接触时,摩擦力的影响可被忽略,以便更便捷地计算其在水平面上的动作。
	
	\item 初始条件固定假定:设定舞龙的初始位置和速度是已知的,并且在整个表演过程中,起始点保持固定不变。
	
	\item  最优路径选择假定:假定舞龙在运动时会选择最理想的路径,以降低能量消耗或提升表演效果。
	
	\item 质量均匀分布假定:假设龙体各节的质量分布均匀,且其质心位置可以精确确定。
	
	\item 空气阻力忽略假定:在舞龙的运动过程中,假设空气阻力的影响可以不计入考量。
	
	\item 连续动作假定:假设舞龙的动作连续不断的,不会出现突然的停顿或中断。
	
	
	
	
	
\end{enumerate}

% ==================================================
% @brief    符号说明
% ==================================================
\clearpage
\mcmSection{符号说明}
对于问题建模和求解的符号说明如下:
\begin{table}[h]
	%[h]表示在此处添加浮动体,默认为tbf,即页面顶部、底部和空白处添加
	\captionsetup{skip=4pt} % 设置标题与表格的间距为4pt
	\caption{符号变量表\label{tab:符号变量表}}
	\centering
	\footnotesize 
	%\scriptsize 
	%\small
	% 对于大表格设置字体为脚注大小或脚本大小或小号字体可能效果更好
	\setlength{\arrayrulewidth}{0.3pt} % 设置表格线条宽度为0.3pt
	\begin{tabular}{ccc}
		% c表示居中,l表示左对齐,r表示右对齐,中间添加“|”表示竖线
		% \hline是横线
		% \Xhline也是横线,但可以用附加选项来控制线宽
		\Xhline{1.5pt}
		% \makebox设置列宽 
		\makebox[0.20\textwidth][c]{符号} & \makebox[0.60\textwidth][c]{说明} & \makebox[0.15\textwidth][c]{单位} \\ 
		\Xhline{0.5pt} 
		$a$ & 螺旋线的起始半径 & cm \\
		$b$ & 螺距常数 (问题1和2中为$\frac{55}{2\pi}$ ,问题3中为待求值) & cm/rad \\
		$r$ & 极径 & cm \\
		$\theta$ & 极角 & rad \\
		$t$ & 时间 & s \\
		$(x, y)$ & 笛卡尔坐标系中的位置 & cm \\
		$(v_x, v_y)$ & 笛卡尔坐标系中的速度 & cm/s \\
		$w$ &   角速度 & cm \\
		$D$ & 调头空间直径 & m \\
		$s$ & 龙头在螺旋线上行进的弧长 & cm \\
		$L_i$ & 第i节板凳的长度 & m \\
		$d_{ij}$ & 第i节板凳与第j节板凳的距离 & m \\
		$p$ & 两圆心之间的距离 & m \\
		
		\Xhline{1.5pt}
	\end{tabular}
	
\end{table}