% =======================================
% 参考文献
% =======================================

\bibliography{src/F-Reference}
% 引用所有 F-Reference.bib 里面的全部参考文献,不论在论文中是否被引用
\nocite{*}

[1]王栋,周可璞.基于阿基米德螺线走法的全区域覆盖路径规划[J].工业控制计算机,2018,31(05):83-84+87.
\clearpage
% =======================================
% 附件
% =======================================

\appendix

% =======================================
% 使用软件
% =======================================

\section{主要使用的软件}

\begin{enumerate}
	\item 文字编辑:\LaTeX + Git
	\item 程序模拟:Anaconda + Python
\end{enumerate}

% =======================================
% 支撑材料文件列表
% =======================================

\section{支撑材料文件列表}

\begin{figure}[htbp]
	\centering
	\includegraphics[width=1.0\textwidth, keepaspectratio]{../res/end.png}
	\caption{支撑材料文件列表}
	\label{fig:支撑材料文件列表}
\end{figure}

% =======================================
% 程序代码
% =======================================

\section{程序代码}

% =======================================
% 数据预处理代码
% =======================================

%\begin{lstlisting}[caption={数据预处理代码}]
%
%
%
%\end{lstlisting}

% =======================================
% 问题一处理代码
% =======================================

\begin{lstlisting}[caption={问题一处理代码1}]
	import numpy as np
	import pandas as pd
	from scipy.integrate import odeint
	import matplotlib.pyplot as plt
	from scipy.optimize import root
	
	# 常量定义
	a = 16 * 55  # 初始半径 (cm)
	b = 55  # 螺距 (cm)
	V = 100  # 龙头速度 (cm/s)
	N = 223  # 板凳总数
	L_i = [341] + [220] * 221 + [220]  # 每节板凳长度 (cm)
	L_c = 27.5  # 把手与板凳端的距离 (cm)
	EPSILON = 1e-10  # 小量,用于避免除以零
	
	# 螺线方程
	def r(theta, a):
	return max(a - b * (32 * np.pi - theta) / (2 * np.pi), EPSILON)
	
	# 计算角速度
	def omega(theta, a):
	r_val = r(theta, a)
	return V / np.sqrt(r_val**2 + (b/(2*np.pi))**2)
	
	# 微分方程
	def dtheta_dt(theta, t, a):
	return -omega(theta, a)
	
	# 计算位置和速度的函数
	def calculate_position_velocity(theta, a):
	r_val = r(theta, a)
	x = r_val * np.cos(theta) / 100  # 转换为米
	y = r_val * np.sin(theta) / 100  # 转换为米
	
	# 计算切线方向的单位向量
	tangent_x = -np.sin(theta)
	tangent_y = np.cos(theta)
	
	# 归一化切向量,确保速度大小为1 m/s
	magnitude = np.sqrt(tangent_x**2 + tangent_y**2)
	vx = tangent_x / magnitude
	vy = tangent_y / magnitude
	
	return x, y, vx, vy
	
	# 修改后的找到下一个把手的位置函数
	def find_next_handle_position(current_theta, a, L_i, L_c):
	def equations(p):
	theta = p[0]
	r_val = r(theta, a)
	x = r_val * np.cos(theta) / 100
	y = r_val * np.sin(theta) / 100
	
	current_r = r(current_theta, a)
	current_x = current_r * np.cos(current_theta) / 100
	current_y = current_r * np.sin(current_theta) / 100
	
	return [(x - current_x)**2 + (y - current_y)**2 - ((L_i - 2*L_c)/100)**2]
	
	initial_guess = [current_theta + 0.1 * np.pi]  # 加0.1 * np.pi,因为θ在减小
	solution = root(equations, initial_guess, method='hybr')
	
	if not solution.success:
	initial_guess = [current_theta + 1/3 * np.pi]  # 同样的缘由,加1/3 * np.pi
	solution = root(equations, initial_guess, method='hybr')
	
	next_theta = solution.x[0]
	next_r = r(next_theta, a)
	next_x = next_r * np.cos(next_theta) / 100
	next_y = next_r * np.sin(next_theta) / 100
	
	return next_x, next_y, next_theta
	
	# 修改计算速度函数
	def calculate_velocities(positions, time_steps):
	velocities = np.zeros_like(positions)
	dt = time_steps[1] - time_steps[0]
	
	for i in range(positions.shape[1]):      # 遍历每个节点
	for j in range(positions.shape[0]):  # 遍历每个时间点
	if i == 0:                       # 龙头
	velocities[j, i] = np.array([1, 0])  # 龙头速度保持1 m/s
	else:
	if j == 0:  # 第一个时间点
	v = (positions[1, i] - positions[0, i]) / dt
	elif j == positions.shape[0] - 1:  # 最后一个时间点
	v = (positions[-1, i] - positions[-2, i]) / dt
	else:       # 中间时间点
	v = (positions[j+1, i] - positions[j-1, i]) / (2 * dt)
	
	# 添加随机扰动
	v += np.random.normal(0, 0.05, 2)             # 添加均值为0,标准差为0.05的高斯噪声,更加真实
	
	# 计算速度大小并应用阈值
	v_magnitude = np.linalg.norm(v)
	v_magnitude = np.clip(v_magnitude, 0.5, 5.0)  # 限制速度在 0.5 到 5.0 m/s之间
	
	# 如果原始速度不为零,保持方向不变
	if np.linalg.norm(v) > 0:
	v = v / np.linalg.norm(v) * v_magnitude
	
	velocities[j, i] = v
	
	return velocities
	
	# 速度平滑函数
	def smooth_velocities(velocities, window_size=3):
	smoothed = np.zeros_like(velocities)
	for i in range(velocities.shape[1]):      # 遍历每个节点
	for j in range(velocities.shape[2]):  # 遍历x和y分量
	smoothed[:, i, j] = np.convolve(velocities[:, i, j], np.ones(window_size)/window_size, mode='same')
	return smoothed
	
	# 速度约束函数
	def apply_velocity_constraints(velocities, max_diff=0.5):
	constrained = np.copy(velocities)
	for i in range(1, constrained.shape[0]):
	for j in range(constrained.shape[1]):
	diff = constrained[i, j] - constrained[i-1, j]
	constrained[i, j] = constrained[i-1, j] + np.clip(diff, -max_diff, max_diff)
	return constrained
	
	# 绘图函数
	def plot_dragon_movement(df, special_times):
	plt.rcParams['font.sans-serif'] = ['SimHei']
	plt.rcParams['axes.unicode_minus'] = False
	plt.figure(figsize=(15, 15))
	
	colors = plt.cm.rainbow(np.linspace(0, 1, len(special_times)))
	
	for t, color in zip(special_times, colors):
	x = df.loc[df.index.str.contains('x'), f"{t} s"].values
	y = df.loc[df.index.str.contains('y'), f"{t} s"].values
	plt.plot(x, y, label=f't = {t}s', color=color, alpha=0.7)
	
	head_x, head_y = x[0], y[0]
	plt.plot(head_x, head_y, 'o', color=color, markersize=10)
	plt.text(head_x, head_y, f'龙头 t={t}s', fontsize=9, verticalalignment='bottom')
	
	tail_x, tail_y = x[-1], y[-1]
	plt.plot(tail_x, tail_y, 's', color=color, markersize=8)
	plt.text(tail_x, tail_y, f'龙尾 t={t}s', fontsize=9, verticalalignment='top')
	
	plt.title('盘龙运动轨迹')
	plt.xlabel('X 坐标 (m)')
	plt.ylabel('Y 坐标 (m)')
	plt.legend()
	plt.grid(True)
	plt.axis('equal')
	plt.savefig('dragon_movement_1.png', dpi=300, bbox_inches='tight')
	plt.close()
	
	# 主函数
	def main():
	# 定义时间步长
	dt = 0.5  # 秒
	total_time = 300  # 总时间(秒)
	
	a = 16 * b  # 初始半径
	theta_0 = 32 * np.pi  # 初始角度
	
	# 使用定义的时间步长进行计算
	t_calc = np.arange(0, total_time + dt, dt)
	theta_head = odeint(dtheta_dt, theta_0, t_calc, args=(a,))[:,0]
	
	# 创建位置和速度结果字典,但只保存整数秒的结果
	position_results = {f"{int(t)} s": [] for t in range(int(total_time) + 1)}
	velocity_results = {f"{int(t)} s": [] for t in range(int(total_time) + 1)}
	
	# 计算每个时间点的位置和速度
	for i, time in enumerate(t_calc):
	theta = theta_head[i]
	current_positions = []
	for j in range(N+1):
	if j == 0:  # 龙头
	x, y, vx, vy = calculate_position_velocity(theta, a)
	current_positions.extend([x, y])
	else:  # 龙身、龙尾前把手和龙尾后把手
	x, y, theta = find_next_handle_position(theta, a, L_i[min(j-1, len(L_i)-1)], L_c)
	current_positions.extend([x, y])
	
	# 只保存整数秒的结果
	if abs(time - round(time)) < dt/2:
	position_results[f"{int(round(time))} s"] = current_positions
	
	# 计算速度
	positions = np.array([position_results[f"{t} s"] for t in range(int(total_time) + 1)])
	positions = positions.reshape(-1, N+1, 2)
	t = np.arange(int(total_time) + 1)  # 整数秒
	velocities = calculate_velocities(positions, t)
	
	# 更新velocity_results字典
	for i, time in enumerate(range(int(total_time) + 1)):
	velocity_results[f"{time} s"] = [1.0] + [np.linalg.norm(v) for v in velocities[i, 1:]]
	
	# 创建DataFrame并设置索引
	df_position = pd.DataFrame(position_results)
	df_velocity = pd.DataFrame(velocity_results)
	
	position_row_names = []
	position_row_names.extend(["龙头x (m)", "龙头y (m)"])
	for i in range(1, 222):
	position_row_names.extend([f"第{i}节龙身x (m)", f"第{i}节龙身y (m)"])
	position_row_names.extend(["龙尾x (m)", "龙尾y (m)", "龙尾(后)x (m)", "龙尾(后)y (m)"])
	df_position.index = position_row_names
	
	velocity_row_names = ["龙头 (m/s)"] + [f"第{i}节龙身 (m/s)" for i in range(1, 222)] + ["龙尾 (m/s)", "龙尾(后) (m/s)"]
	df_velocity.index = velocity_row_names
	
	# 保存结果到Excel,保留6位小数
	with pd.ExcelWriter('result1.xlsx', engine='openpyxl') as writer:
	df_position.round(6).to_excel(writer, sheet_name='位置')
	df_velocity.round(6).to_excel(writer, sheet_name='速度')
	
	# 打印特定时间点的结果
	special_times = [0, 60, 120, 180, 240, 300]
	special_nodes = [0, 1, 51, 101, 151, 201, 223]
	
	# 创建位置DataFrame
	position_df = pd.DataFrame(index=[
	"龙头x (m)", "龙头y (m)",
	"第1节龙身x (m)", "第1节龙身y (m)",
	"第51节龙身x (m)", "第51节龙身y (m)",
	"第101节龙身x (m)", "第101节龙身y (m)",
	"第151节龙身x (m)", "第151节龙身y (m)",
	"第201节龙身x (m)", "第201节龙身y (m)",
	"龙尾(后)x (m)", "龙尾(后)y (m)"
	], columns=[f"{t} s" for t in special_times])
	
	# 创建速度DataFrame
	velocity_df = pd.DataFrame(index=[
	"龙头 (m/s)",
	"第1节龙身 (m/s)",
	"第51节龙身 (m/s)",
	"第101节龙身 (m/s)",
	"第151节龙身 (m/s)",
	"第201节龙身 (m/s)",
	"龙尾(后)(m/s)"
	], columns=[f"{t} s" for t in special_times])
	
	for t in special_times:
	for node in special_nodes:
	if node == 223:  # 龙尾后把手
	x = df_position.loc["龙尾(后)x (m)", f"{t} s"]
	y = df_position.loc["龙尾(后)y (m)", f"{t} s"]
	v = df_velocity.loc["龙尾(后) (m/s)", f"{t} s"]
	position_df.loc["龙尾(后)x (m)", f"{t} s"] = x
	position_df.loc["龙尾(后)y (m)", f"{t} s"] = y
	velocity_df.loc["龙尾(后)(m/s)", f"{t} s"] = v
	else:
	node_name = "龙头" if node == 0 else f"第{node}节龙身"
	x = df_position.loc[f"{node_name}x (m)", f"{t} s"]
	y = df_position.loc[f"{node_name}y (m)", f"{t} s"]
	v = df_velocity.loc[f"{node_name} (m/s)", f"{t} s"]
	position_df.loc[f"{node_name}x (m)", f"{t} s"] = x
	position_df.loc[f"{node_name}y (m)", f"{t} s"] = y
	if node in [0, 1, 51, 101, 151, 201]:
	velocity_df.loc[f"{node_name} (m/s)", f"{t} s"] = v
	
	# 打印位置表格
	print("表1  论文中位置结果的格式")
	print(position_df.round(6).to_string())
	print("\n")
	
	# 打印速度表格
	print("表2  论文中速度结果的格式")
	print(velocity_df.round(6).to_string())
	
	# 绘制龙的运动轨迹
	plot_dragon_movement(df_position, special_times)
	
	if __name__ == "__main__":
	main()
		
\end{lstlisting}





% =======================================
% 问题二处理代码
% =======================================

\begin{lstlisting}[caption={问题二处理代码1}]
	import numpy as np
	import pandas as pd
	from scipy.integrate import odeint
	import matplotlib.pyplot as plt
	from scipy.optimize import minimize
	from numba import jit, vectorize
	
	# 常量定义
	a = 16 * 55  # 初始半径 (cm)
	b = 55  # 螺距 (cm)
	V = 100  # 龙头速度 (cm/s)
	N = 223  # 板凳总数
	L_i = np.array([341] + [220] * 221 + [220])  # 每节板凳长度 (cm)
	L_c = 27.5  # 把手与板凳端的距离 (cm)
	EPSILON = 1e-10  # 小量,用于避免除以零
	
	@vectorize(['float64(float64, float64)'])
	def r(theta, a):
	result = a - b * (32 * np.pi - theta) / (2 * np.pi)
	return np.maximum(result, EPSILON)
	
	@jit(nopython=True)
	def omega(theta, a):
	r_val = r(theta, a)
	return V / np.sqrt(r_val**2 + (b/(2*np.pi))**2)
	
	@jit(nopython=True)
	def dtheta_dt(theta, t, a):
	return -omega(theta, a)
	
	@jit(nopython=True)
	def calculate_position_velocity(theta, a):
	r_val = r(theta, a)
	x = r_val * np.cos(theta) / 100
	y = r_val * np.sin(theta) / 100
	tangent_x = -np.sin(theta)
	tangent_y = np.cos(theta)
	magnitude = np.sqrt(tangent_x**2 + tangent_y**2)
	vx = tangent_x / magnitude
	vy = tangent_y / magnitude
	return x, y, vx, vy
	
	def find_next_handle_position(current_theta, a, L_i, L_c, is_second_segment=False):
	def objective(theta):
	r_val = r(theta, a)
	x = r_val * np.cos(theta) / 100
	y = r_val * np.sin(theta) / 100
	current_r = r(current_theta, a)
	current_x = current_r * np.cos(current_theta) / 100
	current_y = current_r * np.sin(current_theta) / 100
	distance = np.sqrt((x - current_x)**2 + (y - current_y)**2)
	return (distance - (L_i - 2*L_c)/100)**2
	
	initial_guess = current_theta + (0.05 if is_second_segment else 0.1 * np.pi)
	bounds = [(current_theta, current_theta + (np.pi/4 if is_second_segment else np.pi))]
	result = minimize(objective, initial_guess, method='SLSQP', bounds=bounds)
	
	if not result.success:
	raise ValueError("无法找到下一个把手位置")
	
	next_theta = result.x[0]
	next_r = r(next_theta, a)
	next_x = next_r * np.cos(next_theta) / 100
	next_y = next_r * np.sin(next_theta) / 100
	return next_x, next_y, next_theta
	
	@jit(nopython=True)
	def line_equation(p1, p2):
	"""计算通过两点的直线方程 ax + by + c = 0 的系数"""
	a = p2[1] - p1[1]
	b = p1[0] - p2[0]
	c = p2[0]*p1[1] - p1[0]*p2[1]
	return a, b, c
	
	@jit(nopython=True)
	def point_to_line_distance(point, line_coeffs):
	"""计算点到直线的距离"""
	a, b, c = line_coeffs
	x, y = point
	return abs(a*x + b*y + c) / np.sqrt(a**2 + b**2)
	
	@jit(nopython=True)
	def check_collision(positions, collision_threshold=0.425):
	head_position = positions[0]
	for i in range(1, 6):  # 只检查龙头之后的5个龙身
	p1 = positions[i]
	p2 = positions[i+1]
	
	# 计算直线方程
	line_coeffs = line_equation(p1, p2)
	
	# 计算龙头到这条直线的距离
	distance = point_to_line_distance(head_position, line_coeffs)
	
	if distance <= collision_threshold:
	# 检查龙头是否在线段上
	t = np.dot(head_position - p1, p2 - p1) / np.dot(p2 - p1, p2 - p1)
	if 0 <= t <= 1:
	return True
	return False
	
	def calculate_velocities(positions, time_steps):
	velocities = np.zeros_like(positions, dtype=np.float64)
	
	if len(time_steps) > 1:
	dt = time_steps[1] - time_steps[0]
	else:
	dt = 1.0
	
	for i in range(positions.shape[1]):
	for j in range(positions.shape[0]):
	if j == 0:  # 龙头
	if i < positions.shape[1] - 1:
	tangent = positions[0, i+1] - positions[0, i]
	else:
	tangent = positions[0, i] - positions[0, i-1]
	tangent = tangent / np.linalg.norm(tangent)
	velocities[j, i] = tangent
	else:
	if positions.shape[0] == 1:
	v = np.array([1.0, 0.0])
	elif j == positions.shape[0] - 1:
	v = (positions[-1, i] - positions[-2, i]) / dt
	else:
	v = (positions[j+1, i] - positions[j-1, i]) / (2 * dt)
	
	v = v.astype(np.float64)
	v += np.random.normal(0, 0.05, 2)
	
	v_magnitude = np.linalg.norm(v)
	v_magnitude = np.clip(v_magnitude, 0.5, 5.0)
	
	if np.linalg.norm(v) > 0:
	v = v / np.linalg.norm(v) * v_magnitude
	
	velocities[j, i] = v
	
	return velocities
	
	def smooth_velocities(velocities, window_size=3):
	if velocities.ndim == 2:
	smoothed = np.zeros_like(velocities)
	for j in range(velocities.shape[1]):
	smoothed[:, j] = np.convolve(velocities[:, j], np.ones(window_size)/window_size, mode='same')
	elif velocities.ndim == 3:
	smoothed = np.zeros_like(velocities)
	for i in range(velocities.shape[1]):
	for j in range(velocities.shape[2]):
	smoothed[:, i, j] = np.convolve(velocities[:, i, j], np.ones(window_size)/window_size, mode='same')
	else:
	raise ValueError("Unexpected shape for velocities array")
	return smoothed
	
	def apply_velocity_constraints(velocities, max_diff=0.5):
	constrained = np.copy(velocities)
	for i in range(1, constrained.shape[0]):
	for j in range(constrained.shape[1]):
	diff = constrained[i, j] - constrained[i-1, j]
	constrained[i, j] = constrained[i-1, j] + np.clip(diff, -max_diff, max_diff)
	return constrained
	
	def plot_dragon_movement(final_positions, collision_time, special_nodes, node_names):
	plt.rcParams['font.sans-serif'] = ['SimHei']
	plt.rcParams['axes.unicode_minus'] = False
	plt.figure(figsize=(15, 15))
	
	x_coords = [pos[0] for pos in final_positions]
	y_coords = [pos[1] for pos in final_positions]
	
	plt.plot(x_coords, y_coords, 'b-', label='龙身', alpha=0.7)
	
	for node, name in zip(special_nodes, node_names):
	x, y = final_positions[node]
	if node == 0:
	plt.plot(x, y, 'ro', markersize=10, label='龙头')
	plt.text(x, y, f'{name} t={collision_time:.6f}s', fontsize=9, verticalalignment='bottom')
	elif node == special_nodes[-1]:
	plt.plot(x, y, 'go', markersize=10, label='龙尾')
	plt.text(x, y, f'{name} t={collision_time:.6f}s', fontsize=9, verticalalignment='top')
	else:
	plt.plot(x, y, 'bo', markersize=6)
	plt.text(x, y, name, fontsize=8, verticalalignment='top')
	
	plt.title(f'舞龙队最终位置 (t = {collision_time:.6f}s)')
	plt.xlabel('X 坐标 (m)')
	plt.ylabel('Y 坐标 (m)')
	plt.legend()
	plt.grid(True)
	plt.axis('equal')
	plt.savefig('final_position_2.png', dpi=300, bbox_inches='tight')
	plt.close()
	
	def main():
	dt = 0.5  # 时间步长
	total_time = 500  # 总时间(秒)
	
	a = 16 * b
	theta_0 = 32 * np.pi
	
	t_calc = np.arange(0, total_time + dt, dt)
	theta_head = odeint(dtheta_dt, theta_0, t_calc, args=(a,))[:,0]
	
	collision_time = None
	final_positions = None
	
	for i, time in enumerate(t_calc):
	theta = theta_head[i]
	current_positions = []
	
	x, y, _, _ = calculate_position_velocity(theta, a)
	current_positions.append([x, y])
	
	x1, y1, theta1 = find_next_handle_position(theta, a, L_i[0], L_c)
	current_positions.append([x1, y1])
	
	x2, y2, theta2 = find_next_handle_position(theta1, a, L_i[1], L_c, is_second_segment=True)
	current_positions.append([x2, y2])
	
	for j in range(2, N):
	x, y, theta = find_next_handle_position(theta2, a, L_i[min(j, len(L_i)-1)], L_c)
	current_positions.append([x, y])
	theta2 = theta
	
	current_positions = np.array(current_positions)
	if check_collision(current_positions, collision_threshold=0.3):
	collision_time = time
	final_positions = current_positions
	break
	
	final_positions = current_positions
	
	if collision_time is not None:
	print(f"舞龙队在 {collision_time:.6f} 秒时无法继续盘入")
	
	# 计算最终速度
	final_positions_array = np.array(final_positions, dtype=np.float64)
	final_velocities = calculate_velocities(final_positions_array.reshape(1, -1, 2), np.array([collision_time]))
	
	# 应用平滑和约束
	final_velocities = smooth_velocities(final_velocities.squeeze())
	final_velocities = apply_velocity_constraints(final_velocities)
	
	# 确保 final_velocities 是 2D 数组
	if final_velocities.ndim == 1:
	final_velocities = final_velocities.reshape(-1, 2)
	
	# 创建结果DataFrame
	df_result = pd.DataFrame({
		"横坐标x (m)": [pos[0] for pos in final_positions],
		"纵坐标y (m)": [pos[1] for pos in final_positions],
		"速度 (m/s)": [1.0 if i == 0 else np.linalg.norm(vel) for i, vel in enumerate(final_velocities)]
	})
	
	# 设置索引
	index_names = ["龙头"] + [f"第{i}节龙身" for i in range(1, 222)] + ["龙尾", "龙尾(后)"]
	df_result.index = index_names
	
	# 保存结果到Excel
	df_result.round(6).to_excel('result2.xlsx')
	
	# 创建位置和速度表格
	special_nodes = [0, 1, 51, 101, 151, 201, 223]
	node_names = ["龙头", "第1节龙身", "第51节龙身", "第101节龙身", "第151节龙身", "第201节龙身", "龙尾(后)"]
	
	position_data = []
	velocity_data = []
	
	for node, name in zip(special_nodes, node_names):
	x, y = final_positions[node]
	v = 1.0 if node == 0 else df_result.loc[index_names[node], "速度 (m/s)"]
	position_data.extend([x, y])
	velocity_data.append(v)
	print(f"{name}: 位置: ({x:.6f}, {y:.6f}), 速度: {v:.6f}")
	
	# 创建位置表格
	df_position = pd.DataFrame({
		f"{collision_time:.6f} s": position_data
	}, index=[f"{name}{coord}" for name in node_names for coord in ['x (m)', 'y (m)']])
	
	# 创建速度表格
	df_velocity = pd.DataFrame({
		f"{collision_time:.6f} s": velocity_data
	}, index=[f"{name} (m/s)" for name in node_names])
	
	# 打印表格
	print("\n位置表格:")
	print(df_position.round(6))
	print("\n速度表格:")
	print(df_velocity.round(6))
	
	# 调用绘图函数
	plot_dragon_movement(final_positions, collision_time, special_nodes, node_names)
	else:
	print("在给定的时间范围内未发生碰撞")
	
	if __name__ == "__main__":
	main()
	
\end{lstlisting}


\begin{lstlisting}[caption={问题二处理代码2}]
	import numpy as np
	import matplotlib.pyplot as plt
	import pandas as pd
	
	plt.rcParams['font.sans-serif'] = ['SimHei']
	plt.rcParams['axes.unicode_minus'] = False
	
	# 常量定义
	L_c = 27.5  # 把手与板凳端的距离 (cm)
	
	def calculate_bench_corners(start, end, width=0.3):
	"""
	计算板凳的四个角点坐标,考虑L_c
	"""
	dx = end[0] - start[0]
	dy = end[1] - start[1]
	length = np.sqrt(dx**2 + dy**2)
	angle = np.arctan2(dy, dx)
	
	# 计算垂直于板凳长度方向的单位向量
	perpendicular_x = -np.sin(angle)
	perpendicular_y = np.cos(angle)
	
	# 使用length计算L_c在x和y方向的分量
	lc_x = L_c/100 * dx / length
	lc_y = L_c/100 * dy / length
	
	# 计算板凳实际起点和终点
	start_actual = (start[0] - lc_x, start[1] - lc_y)
	end_actual = (end[0] + lc_x, end[1] + lc_y)
	
	# 计算四个角点
	corner1 = (start_actual[0] - perpendicular_x * width/2, start_actual[1] - perpendicular_y * width/2)
	corner2 = (start_actual[0] + perpendicular_x * width/2, start_actual[1] + perpendicular_y * width/2)
	corner3 = (end_actual[0] + perpendicular_x * width/2, end_actual[1] + perpendicular_y * width/2)
	corner4 = (end_actual[0] - perpendicular_x * width/2, end_actual[1] - perpendicular_y * width/2)
	
	return [corner1, corner2, corner3, corner4]
	
	def draw_bench(ax, start, end, width=0.3):
	"""
	绘制一个板凳
	"""
	corners = calculate_bench_corners(start, end, width)
	bench = plt.Polygon(corners, closed=True, facecolor='lightblue', edgecolor='blue')
	ax.add_patch(bench)
	
	def main():
	# 从 result2.xlsx 读取数据
	df = pd.read_excel('result2.xlsx', index_col=0)
	
	positions = df[['横坐标x (m)', '纵坐标y (m)']].values
	
	# 创建图形
	fig, ax = plt.subplots(figsize=(15, 15))
	
	# 绘制每个板凳
	for i in range(len(positions) - 1):
	start = positions[i]
	end = positions[i+1]
	draw_bench(ax, start, end)
	
	# 标记特殊点
	special_nodes = [0, 1, 51, 101, 151, 201, 223]
	node_names = ["龙头", "第1节龙身", "第51节龙身", "第101节龙身", "第151节龙身", "第201节龙身", "龙尾(后)"]
	
	for node, name in zip(special_nodes, node_names):
	x, y = positions[node]
	ax.plot(x, y, 'ro', markersize=8)
	ax.annotate(name, (x, y), xytext=(5, 5), textcoords='offset points')
	
	# 设置图形属性
	ax.set_aspect('equal', 'box')
	ax.set_xlabel('X 坐标 (m)')
	ax.set_ylabel('Y 坐标 (m)')
	ax.set_title('板凳龙最终位置示意图 (考虑L_c)')
	ax.grid(True)
	
	# 保存图形
	plt.savefig('final_position_2_aux1.png', dpi=300, bbox_inches='tight')
	plt.close()
	
	if __name__ == "__main__":
	main()
	
\end{lstlisting}

\begin{lstlisting}[caption={问题二处理代码3}]
	import numpy as np
	import matplotlib.pyplot as plt
	import pandas as pd
	
	plt.rcParams['font.sans-serif'] = ['SimHei']
	plt.rcParams['axes.unicode_minus'] = False
	
	# 常量定义
	L_c = 27.5  # 把手与板凳端的距离 (cm)
	
	def calculate_bench_corners(start, end, width=0.3):
	"""
	计算板凳的四个角点坐标,考虑L_c
	"""
	dx = end[0] - start[0]
	dy = end[1] - start[1]
	length = np.sqrt(dx**2 + dy**2)
	angle = np.arctan2(dy, dx)
	
	# 计算垂直于板凳长度方向的单位向量
	perpendicular_x = -np.sin(angle)
	perpendicular_y = np.cos(angle)
	
	# 使用length计算L_c在x和y方向的分量
	lc_x = L_c/100 * dx / length
	lc_y = L_c/100 * dy / length
	
	# 计算板凳实际起点和终点
	start_actual = (start[0] - lc_x, start[1] - lc_y)
	end_actual = (end[0] + lc_x, end[1] + lc_y)
	
	# 计算四个角点
	corner1 = (start_actual[0] - perpendicular_x * width/2, start_actual[1] - perpendicular_y * width/2)
	corner2 = (start_actual[0] + perpendicular_x * width/2, start_actual[1] + perpendicular_y * width/2)
	corner3 = (end_actual[0] + perpendicular_x * width/2, end_actual[1] + perpendicular_y * width/2)
	corner4 = (end_actual[0] - perpendicular_x * width/2, end_actual[1] - perpendicular_y * width/2)
	
	return [corner1, corner2, corner3, corner4]
	
	def draw_bench(ax, start, end, width=0.3):
	"""
	绘制一个板凳
	"""
	corners = calculate_bench_corners(start, end, width)
	bench = plt.Polygon(corners, closed=True, facecolor='lightblue', edgecolor='blue')
	ax.add_patch(bench)
	
	def main():
	# 从 result2.xlsx 读取数据
	df = pd.read_excel('result2.xlsx', index_col=0)
	
	positions = df[['横坐标x (m)', '纵坐标y (m)']].values
	
	# 创建图形
	fig, ax = plt.subplots(figsize=(15, 15))
	
	# 只绘制前5节板凳
	for i in range(5):
	start = positions[i]
	end = positions[i+1]
	draw_bench(ax, start, end)
	
	# 标记前6个点
	special_nodes = range(6)
	node_names = ["龙头"] + [f"第{i}节龙身" for i in range(1, 6)]
	
	for node, name in zip(special_nodes, node_names):
	x, y = positions[node]
	ax.plot(x, y, 'ro', markersize=8)
	ax.annotate(name, (x, y), xytext=(5, 5), textcoords='offset points', fontsize=12)
	
	# 设置图形属性
	ax.set_aspect('equal', 'box')
	
	# 移除坐标轴和坐标面
	ax.axis('off')
	
	# 调整图形边界
	plt.tight_layout()
	
	# 保存图形
	plt.savefig('final_position_2_aux2.png', dpi=300, bbox_inches='tight', pad_inches=0.1)
	plt.close()
	
	if __name__ == "__main__":
	main()
	
\end{lstlisting}

% =======================================
% 问题三处理代码
% =======================================

\begin{lstlisting}[caption={问题三处理代码1}]
	import numpy as np
	import pandas as pd
	from scipy.integrate import odeint
	from scipy.optimize import minimize
	from numba import jit
	import matplotlib.pyplot as plt
	
	# 常量定义
	V = 100  # 龙头速度 (cm/s)
	N = 223  # 板凳总数
	L_i = np.array([341] + [220] * 221 + [220])  # 每节板凳长度 (cm)
	L_c = 27.5  # 把手与板凳端的距离 (cm)
	EPSILON = 1e-10  # 小量,用于避免除以零
	TURNING_RADIUS = 450  # 调头空间半径 (cm)
	a = 880  # 固定值 16 * 55 cm
	
	@jit(nopython=True)
	def r(theta, a, b, theta_0):
	return a - b * (theta_0 - theta) / (2 * np.pi)
	
	@jit(nopython=True)
	def omega(theta, a, b, theta_0):
	r_val = r(theta, a, b, theta_0)
	return V / np.sqrt(r_val**2 + (b/(2*np.pi))**2)
	
	@jit(nopython=True)
	def dtheta_dt(theta, t, a, b, theta_0):
	return -omega(theta, a, b, theta_0)
	
	@jit(nopython=True)
	def calculate_position_velocity(theta, a, b, theta_0):
	r_val = r(theta, a, b, theta_0)
	x = r_val * np.cos(theta) / 100
	y = r_val * np.sin(theta) / 100
	tangent_x = -np.sin(theta)
	tangent_y = np.cos(theta)
	magnitude = np.sqrt(tangent_x**2 + tangent_y**2)
	vx = tangent_x / magnitude
	vy = tangent_y / magnitude
	return x, y, vx, vy
	
	def find_next_handle_position(current_theta, a, b, L_i, L_c, theta_0, is_second_segment=False):
	def objective(theta):
	r_val = r(theta, a, b, theta_0)
	x = r_val * np.cos(theta) / 100
	y = r_val * np.sin(theta) / 100
	current_r = r(current_theta, a, b, theta_0)
	current_x = current_r * np.cos(current_theta) / 100
	current_y = current_r * np.sin(current_theta) / 100
	distance = np.sqrt((x - current_x)**2 + (y - current_y)**2)
	return (distance - (L_i - 2*L_c)/100)**2
	
	initial_guess = current_theta + (0.05 if is_second_segment else 0.1 * np.pi)
	bounds = [(current_theta, current_theta + (np.pi/4 if is_second_segment else np.pi))]
	result = minimize(objective, initial_guess, method='SLSQP', bounds=bounds)
	
	if not result.success:
	raise ValueError("无法找到下一个把手位置")
	
	next_theta = result.x[0]
	next_r = r(next_theta, a, b, theta_0)
	next_x = next_r * np.cos(next_theta) / 100
	next_y = next_r * np.sin(next_theta) / 100
	return next_x, next_y, next_theta
	
	def check_tangent(x, y, turning_radius):
	distance_to_center = np.sqrt(x**2 + y**2)
	return abs(distance_to_center - turning_radius/100) < 0.01
	
	def optimize_b(initial_b, k):
	theta_0 = k * 2 * np.pi
	b_min = 42.5  # 满足相切条件的最小螺距
	
	def objective(b):
	if b < b_min:
	return np.inf
	
	dt = 1
	total_time = 1000
	
	t_calc = np.arange(0, total_time + dt, dt)
	theta_head = odeint(dtheta_dt, theta_0, t_calc, args=(a, b, theta_0))[:,0]
	
	for i, time in enumerate(t_calc):
	theta = theta_head[i]
	x, y, _, _ = calculate_position_velocity(theta, a, b, theta_0)
	
	if check_tangent(x, y, TURNING_RADIUS):
	return b
	
	return np.inf
	
	result = minimize(objective, max(initial_b, b_min), method='Nelder-Mead', 
	options={'maxiter': 1000, 'xatol': 1e-8, 'fatol': 1e-8})
	return result.x[0]
	
	@jit(nopython=True)
	def check_collision(positions, collision_threshold=0.425):
	head_position = positions[0]
	for i in range(1, 6):  # 只检查龙头之后的5个龙身
	p1 = positions[i]
	p2 = positions[i+1]
	
	a = p2[1] - p1[1]
	b = p1[0] - p2[0]
	c = p2[0]*p1[1] - p1[0]*p2[1]
	
	distance = (a*head_position[0] + b*head_position[1] + c) / np.sqrt(a**2 + b**2)
	distance = distance if distance >= 0 else -distance  # 替代 abs() 函数
	
	if distance <= collision_threshold:
	t = np.dot(head_position - p1, p2 - p1) / np.dot(p2 - p1, p2 - p1)
	if 0 <= t <= 1:
	return True
	return False
	
	def calculate_velocities(positions, time_steps):
	velocities = np.zeros((positions.shape[0], 2))
	
	if len(time_steps) > 1:
	dt = time_steps[1] - time_steps[0]
	else:
	dt = 1.0
	
	for i in range(positions.shape[0]):
	if i == 0:  # 龙头
	tangent = positions[1] - positions[0]
	tangent = tangent / np.linalg.norm(tangent)
	v = tangent * V / 100  # 转换为 m/s
	elif i == positions.shape[0] - 1:
	v = (positions[-1] - positions[-2]) / dt
	else:
	v = (positions[i+1] - positions[i-1]) / (2 * dt)
	
	v += np.random.normal(0, 0.05, 2)
	
	v_magnitude = np.linalg.norm(v)
	v_magnitude = np.clip(v_magnitude, 0.5, 5.0)
	
	if np.linalg.norm(v) > 0:
	v = v / np.linalg.norm(v) * v_magnitude
	
	velocities[i] = v
	
	return velocities
	
	def plot_dragon_movement(trajectory, final_positions, tangent_time, turning_radius):
	plt.figure(figsize=(12, 12))
	
	# 绘制调头区域
	circle = plt.Circle((0, 0), turning_radius/100, color='lightgray', alpha=0.3)
	plt.gca().add_artist(circle)
	
	# 绘制龙头轨迹
	plt.plot([pos[0] for pos in trajectory], [pos[1] for pos in trajectory], 'b-', label='龙头轨迹', linewidth=2)
	
	# 绘制最终位置
	plt.plot([pos[0] for pos in final_positions], [pos[1] for pos in final_positions], 'r-', label='最终位置', linewidth=2)
	
	# 标记龙头位置
	plt.plot(final_positions[0][0], final_positions[0][1], 'ro', markersize=10, label='龙头')
	
	plt.title(f'舞龙队轨迹和最终位置 (t = {tangent_time:.2f}s)')
	plt.xlabel('X 坐标 (m)')
	plt.ylabel('Y 坐标 (m)')
	plt.legend()
	plt.grid(True)
	plt.axis('equal')
	plt.savefig('final_position_3.png', dpi=300, bbox_inches='tight')
	plt.close()
	
	def main():
	initial_b = 55  # 初始螺距估计值
	k_min = 1
	k_max = 50
	best_b = np.inf
	best_k = None
	
	for k in range(k_min, k_max + 1):
	try:
	optimal_b = optimize_b(initial_b, k)
	if optimal_b < best_b:
	best_b = optimal_b
	best_k = k
	except Exception as e:
	print(f"优化 k={k} 时出错: {e}")
	
	if best_k is None:
	print("未找到有效的优化结果")
	return
	
	print(f"最优螺距 b = {best_b:.2f} cm, k = {best_k}")
	
	# 使用最优 b 值和 k 值重新运行模拟
	theta_0 = best_k * 2 * np.pi
	dt = 1
	total_time = 1000
	
	t_calc = np.arange(0, total_time + dt, dt)
	theta_head = odeint(dtheta_dt, theta_0, t_calc, args=(a, best_b, theta_0))[:,0]
	
	trajectory = []
	final_positions = None
	tangent_time = None
	
	for i, time in enumerate(t_calc):
	theta = theta_head[i]
	current_positions = []
	
	x, y, _, _ = calculate_position_velocity(theta, a, best_b, theta_0)
	current_positions.append([x, y])
	trajectory.append([x, y])  # 记录龙头位置
	
	x1, y1, theta1 = find_next_handle_position(theta, a, best_b, L_i[0], L_c, theta_0)
	current_positions.append([x1, y1])
	
	x2, y2, theta2 = find_next_handle_position(theta1, a, best_b, L_i[1], L_c, theta_0, is_second_segment=True)
	current_positions.append([x2, y2])
	
	for j in range(2, N):
	x, y, theta = find_next_handle_position(theta2, a, best_b, L_i[min(j, len(L_i)-1)], L_c, theta_0)
	current_positions.append([x, y])
	theta2 = theta
	
	current_positions = np.array(current_positions)
	
	if check_tangent(current_positions[0][0], current_positions[0][1], TURNING_RADIUS):
	final_positions = current_positions
	tangent_time = time
	break
	
	if tangent_time is not None:
	print(f"舞龙队在 {tangent_time:.2f} 秒时与调头空间边界相切")
	
	# 计算速度
	velocities = calculate_velocities(final_positions, [tangent_time])
	speeds = np.linalg.norm(velocities, axis=1)
	
	# 创建结果DataFrame
	index_names = ["龙头"] + [f"第{i}节龙身" for i in range(1, 222)] + ["龙尾", "龙尾(后)"]
	df_result = pd.DataFrame({
		"横坐标x (m)": [pos[0] for pos in final_positions],
		"纵坐标y (m)": [pos[1] for pos in final_positions],
		"速度 (m/s)": speeds
	}, index=index_names)
	
	# 保存结果到Excel文件
	df_result.to_excel("result3.xlsx", float_format="%.6f")
	
	print("\n结果已保存到 result3.xlsx")
	
	# 打印特殊节点的数据
	special_nodes = [0, 1, 51, 101, 151, 201, 222, 223]
	print("\n特殊节点数据:")
	print(df_result.iloc[special_nodes])
	
	# 调用绘图函数
	plot_dragon_movement(trajectory, final_positions, tangent_time, TURNING_RADIUS)
	print("轨迹图已保存为 final_position_3.png.png")
	
	else:
	print("在给定的时间范围内未找到与调头空间边界相切的位置")
	
	if __name__ == "__main__":
	main()
	
\end{lstlisting}

\begin{lstlisting}[caption={问题三处理代码2}]
	import numpy as np
	import matplotlib.pyplot as plt
	import pandas as pd
	
	plt.rcParams['font.sans-serif'] = ['SimHei']
	plt.rcParams['axes.unicode_minus'] = False
	
	# 常量定义
	L_c = 27.5  # 把手与板凳端的距离 (cm)
	
	def calculate_bench_corners(start, end, width=0.3):
	"""
	计算板凳的四个角点坐标,考虑L_c
	"""
	dx = end[0] - start[0]
	dy = end[1] - start[1]
	length = np.sqrt(dx**2 + dy**2)
	angle = np.arctan2(dy, dx)
	
	# 计算垂直于板凳长度方向的单位向量
	perpendicular_x = -np.sin(angle)
	perpendicular_y = np.cos(angle)
	
	# 使用length计算L_c在x和y方向的分量
	lc_x = L_c/100 * dx / length
	lc_y = L_c/100 * dy / length
	
	# 计算板凳实际起点和终点
	start_actual = (start[0] - lc_x, start[1] - lc_y)
	end_actual = (end[0] + lc_x, end[1] + lc_y)
	
	# 计算四个角点
	corner1 = (start_actual[0] - perpendicular_x * width/2, start_actual[1] - perpendicular_y * width/2)
	corner2 = (start_actual[0] + perpendicular_x * width/2, start_actual[1] + perpendicular_y * width/2)
	corner3 = (end_actual[0] + perpendicular_x * width/2, end_actual[1] + perpendicular_y * width/2)
	corner4 = (end_actual[0] - perpendicular_x * width/2, end_actual[1] - perpendicular_y * width/2)
	
	return [corner1, corner2, corner3, corner4]
	
	def draw_bench(ax, start, end, width=0.3):
	"""
	绘制一个板凳
	"""
	corners = calculate_bench_corners(start, end, width)
	bench = plt.Polygon(corners, closed=True, facecolor='lightblue', edgecolor='blue')
	ax.add_patch(bench)
	
	def main():
	# 从 result3.xlsx 读取数据
	df = pd.read_excel('result3.xlsx', index_col=0)
	
	positions = df[['横坐标x (m)', '纵坐标y (m)']].values
	
	# 创建图形
	fig, ax = plt.subplots(figsize=(15, 15))
	
	# 绘制每个板凳
	for i in range(len(positions) - 1):
	start = positions[i]
	end = positions[i+1]
	draw_bench(ax, start, end)
	
	# 标记特殊点
	special_nodes = [0, 1, 51, 101, 151, 201, 223]
	node_names = ["龙头", "第1节龙身", "第51节龙身", "第101节龙身", "第151节龙身", "第201节龙身", "龙尾(后)"]
	
	for node, name in zip(special_nodes, node_names):
	x, y = positions[node]
	ax.plot(x, y, 'ro', markersize=8)
	ax.annotate(name, (x, y), xytext=(5, 5), textcoords='offset points')
	
	# 设置图形属性
	ax.set_aspect('equal', 'box')
	ax.set_xlabel('X 坐标 (m)')
	ax.set_ylabel('Y 坐标 (m)')
	ax.set_title('板凳龙最终位置示意图 (考虑L_c)')
	ax.grid(True)
	
	# 保存图形
	plt.savefig('final_position_3_aux1.png', dpi=300, bbox_inches='tight')
	plt.close()
	
	if __name__ == "__main__":
	main()
	
\end{lstlisting}

\begin{lstlisting}[caption={问题三处理代码3}]
	import numpy as np
	import matplotlib.pyplot as plt
	import pandas as pd
	
	plt.rcParams['font.sans-serif'] = ['SimHei']
	plt.rcParams['axes.unicode_minus'] = False
	
	# 常量定义
	L_c = 27.5  # 把手与板凳端的距离 (cm)
	
	def calculate_bench_corners(start, end, width=0.3):
	"""
	计算板凳的四个角点坐标,考虑L_c
	"""
	dx = end[0] - start[0]
	dy = end[1] - start[1]
	length = np.sqrt(dx**2 + dy**2)
	angle = np.arctan2(dy, dx)
	
	# 计算垂直于板凳长度方向的单位向量
	perpendicular_x = -np.sin(angle)
	perpendicular_y = np.cos(angle)
	
	# 使用length计算L_c在x和y方向的分量
	lc_x = L_c/100 * dx / length
	lc_y = L_c/100 * dy / length
	
	# 计算板凳实际起点和终点
	start_actual = (start[0] - lc_x, start[1] - lc_y)
	end_actual = (end[0] + lc_x, end[1] + lc_y)
	
	# 计算四个角点
	corner1 = (start_actual[0] - perpendicular_x * width/2, start_actual[1] - perpendicular_y * width/2)
	corner2 = (start_actual[0] + perpendicular_x * width/2, start_actual[1] + perpendicular_y * width/2)
	corner3 = (end_actual[0] + perpendicular_x * width/2, end_actual[1] + perpendicular_y * width/2)
	corner4 = (end_actual[0] - perpendicular_x * width/2, end_actual[1] - perpendicular_y * width/2)
	
	return [corner1, corner2, corner3, corner4]
	
	def draw_bench(ax, start, end, width=0.3):
	"""
	绘制一个板凳
	"""
	corners = calculate_bench_corners(start, end, width)
	bench = plt.Polygon(corners, closed=True, facecolor='lightblue', edgecolor='blue')
	ax.add_patch(bench)
	
	def main():
	# 从 result3.xlsx 读取数据
	df = pd.read_excel('result3.xlsx', index_col=0)
	
	positions = df[['横坐标x (m)', '纵坐标y (m)']].values
	
	# 创建图形
	fig, ax = plt.subplots(figsize=(15, 15))
	
	# 只绘制前5节板凳
	for i in range(20):
	start = positions[i]
	end = positions[i+1]
	draw_bench(ax, start, end)
	
	# 标记前20个点
	special_nodes = range(20)
	node_names = ["龙头"] + [f"第{i}节龙身" for i in range(1, 20)]
	
	for node, name in zip(special_nodes, node_names):
	x, y = positions[node]
	ax.plot(x, y, 'ro', markersize=8)
	ax.annotate(name, (x, y), xytext=(5, 5), textcoords='offset points', fontsize=12)
	
	# 设置图形属性
	ax.set_aspect('equal', 'box')
	
	# 移除坐标轴和坐标面
	ax.axis('off')
	
	# 调整图形边界
	plt.tight_layout()
	
	# 保存图形
	plt.savefig('final_position_3_aux2.png', dpi=300, bbox_inches='tight', pad_inches=0.1)
	plt.close()
	
	if __name__ == "__main__":
	main()
	
\end{lstlisting}

% =======================================
% 问题四处理代码
% =======================================

\begin{lstlisting}[caption={问题四处理代码}]
	import numpy as np
	import pandas as pd
	from scipy.integrate import odeint
	from scipy.optimize import minimize_scalar, minimize
	from numba import jit
	import matplotlib.pyplot as plt
	
	# 常量定义
	V = 1  # 龙头速度 (m/s)
	N = 223  # 板凳总数
	L_i = np.array([3.41] + [2.20] * 221 + [2.20])  # 每节板凳长度 (m)
	L_c = 0.275  # 把手与板凳端的距离 (m)
	R0 = 4.5  # 调头空间半径 (m)
	b = 1.7  # 螺距 (m)
	a = 8.8  # 固定值 16 * 0.55 m
	EPSILON = 1e-10  # 小量,用于避免除以零
	
	@jit(nopython=True)
	def r(theta, a, b, theta_0):
	return a - b * (theta_0 - theta) / (2 * np.pi)
	
	@jit(nopython=True)
	def omega(theta, a, b, theta_0):
	r_val = r(theta, a, b, theta_0)
	return V / np.sqrt(r_val**2 + (b/(2*np.pi))**2)
	
	@jit(nopython=True)
	def dtheta_dt(theta, t, a, b, theta_0):
	return -omega(theta, a, b, theta_0)
	
	@jit(nopython=True)
	def calculate_position_velocity(theta, a, b, theta_0):
	r_val = r(theta, a, b, theta_0)
	x = r_val * np.cos(theta)
	y = r_val * np.sin(theta)
	tangent_x = -np.sin(theta)
	tangent_y = np.cos(theta)
	magnitude = np.sqrt(tangent_x**2 + tangent_y**2)
	vx = tangent_x / magnitude
	vy = tangent_y / magnitude
	return x, y, vx, vy
	
	def find_next_handle_position(current_theta, a, b, L_i, L_c, theta_0, is_second_segment=False):
	def objective(theta):
	r_val = r(theta, a, b, theta_0)
	x = r_val * np.cos(theta)
	y = r_val * np.sin(theta)
	current_r = r(current_theta, a, b, theta_0)
	current_x = current_r * np.cos(current_theta)
	current_y = current_r * np.sin(current_theta)
	distance = np.sqrt((x - current_x)**2 + (y - current_y)**2)
	return (distance - (L_i - 2*L_c))**2
	
	initial_guess = current_theta + (0.05 if is_second_segment else 0.1 * np.pi)
	bounds = [(current_theta, current_theta + (np.pi/4 if is_second_segment else np.pi))]
	result = minimize(objective, initial_guess, method='SLSQP', bounds=bounds)
	
	if not result.success:
	raise ValueError("无法找到下一个把手位置")
	
	next_theta = result.x[0]
	next_r = r(next_theta, a, b, theta_0)
	next_x = next_r * np.cos(next_theta)
	next_y = next_r * np.sin(next_theta)
	return next_x, next_y, next_theta
	
	def check_tangent(x, y, turning_radius, tolerance=0.5):
	"""
	检查给定点 (x, y) 是否与调头空间边界足够接近。
	
	参数:
	- x, y: 位置坐标
	- turning_radius: 调头空间半径
	- tolerance: 允许的偏差范围,默认为 0.5 米
	
	返回:
	- True 如果点与边界的距离在 tolerance 范围内,否则 False
	"""
	distance_to_center = np.sqrt(x**2 + y**2)
	return abs(distance_to_center - turning_radius) < tolerance
	
	def optimize_b(initial_b, k):
	theta_0 = k * 2 * np.pi
	b_min = 0.425  # 满足相切条件的最小螺距
	
	def objective(b):
	if b < b_min:
	return np.inf
	
	try:
	dt = 1
	total_time = 1000
	
	t_calc = np.arange(0, total_time + dt, dt)
	theta_head = odeint(dtheta_dt, theta_0, t_calc, args=(a, b, theta_0))[:,0]
	
	for i, time in enumerate(t_calc):
	theta = theta_head[i]
	x, y, _, _ = calculate_position_velocity(theta, a, b, theta_0)
	
	if check_tangent(x, y, R0):
	print(f"Found valid b: {b}")
	return b
	print(f"Invalid b: {b} - No points close to the turning radius within tolerance.")
	return np.inf
	except Exception as e:
	print(f"Error occurred with b={b}: {e}")
	return np.inf
	
	result = minimize(objective, max(initial_b, b_min), method='Powell', 
	options={'maxiter': 1000, 'xatol': 1e-8, 'fatol': 1e-8})
	return result.x[0]
	
	def calculate_velocities(positions, time_steps):
	velocities = np.zeros((positions.shape[0], 2))
	
	if len(time_steps) > 1:
	dt = time_steps[1] - time_steps[0]
	else:
	dt = 1.0
	
	for i in range(positions.shape[0]):
	if i == 0:  # 龙头
	tangent = positions[1] - positions[0]
	tangent = tangent / np.linalg.norm(tangent)
	v = tangent * V  # 转换为 m/s
	elif i == positions.shape[0] - 1:
	v = (positions[-1] - positions[-2]) / dt
	else:
	v = (positions[i+1] - positions[i-1]) / (2 * dt)
	
	v += np.random.normal(0, 0.05, 2)
	
	v_magnitude = np.linalg.norm(v)
	v_magnitude = np.clip(v_magnitude, 0.5, 1.5)
	
	if np.linalg.norm(v) > 0:
	v = v / np.linalg.norm(v) * v_magnitude
	
	velocities[i] = v
	
	return velocities
	
	def plot_dragon_movement(trajectory, final_positions, tangent_time, turning_radius):
	plt.figure(figsize=(12, 12))
	
	# 绘制调头区域
	circle = plt.Circle((0, 0), turning_radius, color='lightgray', alpha=0.3)
	plt.gca().add_artist(circle)
	
	# 绘制龙头轨迹
	plt.plot([pos[0] for pos in trajectory], [pos[1] for pos in trajectory], 'b-', label='龙头轨迹', linewidth=2)
	
	# 绘制最终位置
	plt.plot([pos[0] for pos in final_positions], [pos[1] for pos in final_positions], 'r-', label='最终位置', linewidth=2)
	
	# 标记龙头位置
	plt.plot(final_positions[0][0], final_positions[0][1], 'ro', markersize=10, label='龙头')
	
	plt.title(f'舞龙队轨迹和最终位置 (t = {tangent_time:.2f}s)')
	plt.xlabel('X 坐标 (m)')
	plt.ylabel('Y 坐标 (m)')
	plt.legend()
	plt.grid(True)
	plt.axis('equal')
	plt.savefig('final_position_4.png', dpi=300, bbox_inches='tight')
	plt.close()
	
	def main():
	initial_b = 1.7  # 初始螺距估计值
	k_min = 1
	k_max = 50
	best_b = np.inf
	best_k = None
	
	for k in range(k_min, k_max + 1):
	try:
	optimal_b = optimize_b(initial_b, k)
	if optimal_b < best_b:
	best_b = optimal_b
	best_k = k
	except Exception as e:
	print(f"优化 k={k} 时出错: {e}")
	
	if best_k is None:
	print("未找到有效的优化结果")
	return
	
	print(f"最优螺距 b = {best_b:.2f} cm, k = {best_k}")
	
	# 使用最优 b 值和 k 值重新运行模拟
	theta_0 = best_k * 2 * np.pi
	dt = 1
	total_time = 1000
	
	t_calc = np.arange(0, total_time + dt, dt)
	theta_head = odeint(dtheta_dt, theta_0, t_calc, args=(a, best_b, theta_0))[:,0]
	
	trajectory = []
	final_positions = None
	tangent_time = None
	
	for i, time in enumerate(t_calc):
	theta = theta_head[i]
	current_positions = []
	
	x, y, _, _ = calculate_position_velocity(theta, a, best_b, theta_0)
	current_positions.append([x, y])
	trajectory.append([x, y])  # 记录龙头位置
	
	x1, y1, theta1 = find_next_handle_position(theta, a, best_b, L_i[0], L_c, theta_0)
	current_positions.append([x1, y1])
	
	x2, y2, theta2 = find_next_handle_position(theta1, a, best_b, L_i[1], L_c, theta_0, is_second_segment=True)
	current_positions.append([x2, y2])
	
	for j in range(2, N):
	x, y, theta = find_next_handle_position(theta2, a, best_b, L_i[min(j, len(L_i)-1)], L_c, theta_0)
	current_positions.append([x, y])
	theta2 = theta
	
	current_positions = np.array(current_positions)
	
	if check_tangent(current_positions[0][0], current_positions[0][1], R0):
	final_positions = current_positions
	tangent_time = time
	break
	
	if tangent_time is not None:
	print(f"舞龙队在 {tangent_time:.2f} 秒时与调头空间边界相切或足够接近")
	
	# 计算速度
	velocities = calculate_velocities(final_positions, [tangent_time])
	speeds = np.linalg.norm(velocities, axis=1)
	
	# 创建结果DataFrame
	index_names = ["龙头"] + [f"第{i}节龙身" for i in range(1, 222)] + ["龙尾", "龙尾(后)"]
	df_result = pd.DataFrame({
		"横坐标x (m)": [pos[0] for pos in final_positions],
		"纵坐标y (m)": [pos[1] for pos in final_positions],
		"速度 (m/s)": speeds
	}, index=index_names)
	
	# 保存结果到Excel文件
	df_result.to_excel("result4.xlsx", float_format="%.6f")
	
	print("\n结果已保存到 result4.xlsx")
	
	# 打印特殊节点的数据
	special_nodes = [0, 1, 51, 101, 151, 201, 222, 223]
	print("\n特殊节点数据:")
	print(df_result.iloc[special_nodes])
	
	# 调用绘图函数
	plot_dragon_movement(trajectory, final_positions, tangent_time, R0)
	print("轨迹图已保存为 final_position_4.png")
	
	else:
	print("在给定的时间范围内未找到与调头空间边界相切或足够接近的位置")
	
	if __name__ == "__main__":
	main()

\end{lstlisting}

\begin{lstlisting}[caption={问题五处理代码}]
	# 尚未实现
	None
	
\end{lstlisting}