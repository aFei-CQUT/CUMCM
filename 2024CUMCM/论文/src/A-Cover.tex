\thispagestyle{empty}   % 定义起始页的页眉页脚格式为 empty —— 空,没有页眉页脚

\begin{center}
	\textbf{\fontsize{20}{1.5}基于阿基米德等距螺线的板凳龙数学模型}
	
	
	\textbf{摘 要}
\end{center}

% ==================================================
% @brief    论文摘要
% ==================================================
% 这里写摘要内容
通过建立螺线方程的数学模型,解决了舞龙队行进过程中沿螺线盘入、调头和盘出的运动问题。


\textbf {针对问题一:}
首先简要介绍几种\textbf{等距螺旋线的基本表达形式},以等距螺旋线中心为坐标原点,在极坐标系下建立板凳龙的行进轨迹方程,并通过极坐标系与笛卡尔坐标系之间的转换关系,通过将笛卡尔坐标系转换至极坐标系,能够计算出不同时间点龙头的极坐标位置。导出\textbf{直角坐标系下龙头,龙身,龙尾的行进轨迹方程}$\mathbf{r(t)=(x(t),y(t))}$\textbf{及其对应的速度方程}$V(t)=d_r(t)/dt$。考虑到龙体的长度,先对坐标进行了相应的位移调整,再确定龙体各部分的准确位置。运用链式法则和微分运算获得了龙体各部分的速度。最后将极坐标位置转换回笛卡尔坐标系,以得到更为直观的位置数据。


\textbf {针对问题二:}
基于问题一建立的螺旋模型,计算得出龙体各部分的位移数据。然后再采用迭代方法分析了板凳间的间距变化,从而确定舞龙队在不发生碰撞情况下的盘入终止时间,并将相关数据存入result2.xlsx文件。结合欧几里得距离的计算,通过设定约束条件$d_{i,i+1}(t)<=0.3$,并采用时间步进迭代法来求解欧几里得距离的临界值,\textbf{即理论上最短距离应为42.5cm}。根据约束条件来决定是否停止迭代,最后利用位移信息计算得出龙体各部分的速度。

\textbf {针对问题三:}
先构建了一个满足最小螺距方程,以确保龙头能够顺利进入预定调头区域。通过应用梯度下降进行迭代求解进行多代的选择和遗传,然后通过监控适应度值的变化来评估算法的性能。根据最优适应度,确定了最佳螺距值。最后在模型的优化中给出了更合适的解法。


\textbf {针对问题四:}
首先构建调头曲线的轨迹,对S形调头曲线的优化,成功缩短了调头路径,并记录了调头过程中的位置与速度变化至result4.xlsx文件。在传统的优化方法基础上,采用前馈神经网络进行迭代求解。通过大量数据的学习,能够判断通过调整圆弧曲线可以改变调头曲线的长度。借鉴前面三问的解决办法,建立板凳龙螺旋运动轨迹的方程,最后求解出龙体各部分的位置和速度。


\textbf {针对问题五:}
依据前文提到的圆弧曲线上的不同位置与龙头速度的关系,推断出在半径最小处,龙体速度达到最大。通过速度分析,确定了龙头在保证舞龙队各部分速度不超过2 m/s的前提下的最大行进速度。基于这一条件,成功反向推导出了龙头的最大速度。

{\textbf{关键词:}  % 3~5个,最多5个
	 {梯度下降算法} \quad   
	 {链式法则}     \quad   
	 {极坐标转换}   \quad   
	 {递推迭代公式} \quad
	 {优化模型} \quad
	%\textbf{} \quad 
	%\textbf{} \quad  
	%\textbf{} \quad 
	%\textbf{} % end
