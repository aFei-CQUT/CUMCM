% =======================================
% @brief 一些常用的语法结构放在本文档
% =======================================


% =======================================
% 方程格式
% =======================================

%\begin{equation}
%	M2_t = \gamma_0 + \sum_{i=1}^{p} \gamma_i M2_{t-i} + \sum_{j=1}^{q} \delta_j GNI_{t-j} + \eta_t
%\end{equation}





% =======================================
% 正文中带有数学符号时
% =======================================

% $( \pi_t )$ 是时间 ( t ) 的通货膨胀率。





% =======================================
% 符号说明表格
% =======================================

%\begin{tabular}{c|c|c}
%	\hline 符号 & 说明 & 单位 \\
%	\hline \hline
%	$W$ & 多波束测量覆盖宽度 & $m$ \\
%	$\theta$ & 多波束换能器开角 & $\circ$ \\
%	$\alpha$ & 海底坡度 & $\circ$ \\
%	$D$ & 海水深度 & $m$ \\
%	$\eta$ & 相邻条带之间重叠率 & $\%$ \\
%	$D_0$ & 问题一海域中心处海水深度 & $m$ \\
%	$\beta$ & 测线方向与海底坡面的法向在水平面上投影的夹角 & $\circ$ \\
%	$\delta$ & 问题二水平面与覆盖宽度所在直线的夹角 & $\circ$ \\
%	$H_0$ & 问题二海域中心处海水深度 & $m$ \\
%	$L$ & 问题二测量船距海域中心处的距离 & $m$ \\
%	$I_1$ & 矩形海域南北长度 & $m$ \\
%	$I_2$ & 矩形海域东西宽度 & $m$ \\
%	\hline
%\end{tabular}





% =======================================
% 正文表格
% =======================================

%\begin{table}[htbp]
%	\centering
%	\caption{统计量数据}
%	\footnotesize % 对于大表格设置字体为脚注大小或脚本大小或小字体可能效果更好 \scriptsize \small
%	\begin{tabular}{@{}lrrrr@{}}
	%		\toprule
	%		统计量名称 & M2(单位:元) & GNI(单位:元) & IR(单位:百分号) & RIR(单位:百分号) \\ \midrule
	%		数据个数 & 30 & 30 & 30 & 30 \\
	%		平均值 & $8.9239 \times 10^{13}$ & $5.3023 \times 10^{13}$ & $3.34$ & $2.24$ \\
	%		标准差 & $8.6307 \times 10^{13}$ & $3.4811 \times 10^{13}$ & $5.21$ & $3.29$ \\
	%		最小值 & $4.6920 \times 10^{12}$ & $1.1446 \times 10^{13}$ & $-1.40$ & $-7.99$ \\
	%		四分之一分位数 & $1.6155 \times 10^{13}$ & $2.1286 \times 10^{13}$ & $1.02$ & $-0.12$ \\
	%		四分之二分位数 & $5.4270 \times 10^{13}$ & $4.5314 \times 10^{13}$ & $1.99$ & $2.96$ \\
	%		四分之三分位数 & $1.5106 \times 10^{14}$ & $8.0646 \times 10^{13}$ & $3.11$ & $4.22$ \\
	%		最大值 & $2.8734 \times 10^{14}$ & $1.1909 \times 10^{14}$ & $24.26$ & $7.36$ \\
	%		偏度 & $8.7368 \times 10^{-1}$ & $4.8477 \times 10^{-1}$ & $2.95$ & $-0.95$ \\
	%		峰度 & $-4.4214 \times 10^{-1}$ & $-1.1427 \times 10^{-1}$ & $9.70$ & $1.79$ \\ 
	%		\bottomrule
	%	\end{tabular}
%	\label{tab:statistics}
%\end{table}





% =======================================
% 分左右两块界面插入图片并描述(一图一说明)
% =======================================

%\begin{center}
%	\begin{figure}[h]
%		\centering
%		\begin{minipage}[t]{0.6\textwidth}
%			\centering
%			\dashbox{\includegraphics[width=\textwidth]{../res/1.png}} % 使用虚线框住图片
%			\caption{盘入螺线示意图}
%			\label{fig:LuoXuanXian}
%		\end{minipage}%
%		\hspace*{1cm} % 插入水平空间
%		\begin{minipage}[t]{0.34\textwidth}
%			\raggedright % 左对齐
%			\vspace{-5.4cm} % 调整此值以实现更好的对齐文字顶端和图形顶端
%			\qquad 在问题一中,已知初始时,龙头位于螺线第16圈A点处(见图1),等距螺旋线螺距b,龙头前把手中心的行进速度,要求我们建立能刻画盘龙队行进轨迹的等距螺旋线模型(内旋)。请给出从初始时刻到300 s为止,每秒整个舞龙队把手中心沿等距螺旋模型的位置、速度,所有计算结果以题目所附表格的形式存储到result1.xlsx中,并在论文以表1,表二的格式,列出0 s、60 s、120 s、180 s、240 s、300 s时,龙头前把手、龙头后面第1、51、101、151、201节龙身前把手和龙尾后把手的位置和速度。
%		\end{minipage}
%	\end{figure}
%\end{center}





% =======================================
% 分左右两块界面插入图片并描述(三图三说明)
% =======================================

%\begin{figure}[H]
%	\centering
%	\begin{minipage}[t]{0.48\linewidth}
%		\subfloat[Bar Plot]{%
%			\includegraphics[width=\linewidth]{../res/1_bar_plot.png}%
%			\label{fig:bar_plot}%
%		}
%		
%		\vspace{0.5em}
%		
%		\subfloat[Box Plot]{%
%			\includegraphics[width=\linewidth]{../res/2_box_plot.png}%
%			\label{fig:box_plot}%
%		}
%		
%		\vspace{0.5em}
%		
%		\subfloat[Area Plot]{%
%			\includegraphics[width=\linewidth]{../res/3_area_plot.png}%
%			\label{fig:area_plot}%
%		}
%	\end{minipage}%
%	\hfill
%	\begin{minipage}[t]{0.48\linewidth}
%		\vspace{1em}
%		\textbf{图 (a) 描述:}\\
%		子图(a)绘出了1994年以后30年的M2,GNI,RIR(实际利率,Real Intrest Rate),IR(以CPI为指数的通货膨胀率, Inflation Rate - CPI)的条形统计图。由图可得,M2和GNI指标逐年递增,而RIR和IR有正有负,与当年的经济状况,国际形势以及意外事故有关。
%		
%		\vspace{4em}
%		
%		\textbf{图 (b) 描述:}\\
%		通过对子图(b)的观察我们可以看到各个指标$\frac{1}{4}$,$\frac{2}{4}$,$\frac{3}{4}$,$\frac{4}{4}$的分位情况,其中,M2及GNI数据分布空间较大,而IR与RIR分布区间较小。需要注意的是,尽管各个指标的量纲不同,但是通过箱线图反映各个指标自身数据四分之比例区间也是具有可比性的。
%		
%		\vspace{4em}
%		
%		\textbf{图 (c) 描述:}\\
%		子图(c)既显示了各个数据随年份的变化趋势也显示了各个指标累计加和的变化情况,各指标的变化趋势实质上与条形图反映一致,不再赘述。而加和情况反映了多年来各个指标的总效益,由图易知,各个指标在时间上纵向求和均为正值,即各个指标在时间维度上加和具有单增趋势。
%	\end{minipage}
%	
%	\caption{不同类型的图及其描述}
%	\label{fig:all_plots}
%\end{figure}





% =======================================
% 算法示例
% =======================================

%\begin{algorithm}  
%	\caption{用归并排序求逆序数}  
%	\begin{algorithmic}[1] %每行显示行号  
%		\Require $Array$数组,$n$数组大小  
%		\Ensure 逆序数  
%		\Function {MergerSort}{$Array, left, right$}  
%		\State $result \gets 0$  
%		\If {$left < right$}  
%		\State $middle \gets (left + right) / 2$  
%		\State $result \gets result +$ \Call{MergerSort}{$Array, left, middle$}  
%		\State $result \gets result +$ \Call{MergerSort}{$Array, middle, right$}  
%		\State $result \gets result +$ \Call{Merger}{$Array,left,middle,right$}  
%		\EndIf  
%		\State \Return{$result$}  
%		\EndFunction  
%		\State  
%		\Function{Merger}{$Array, left, middle, right$}  
%		\State $i\gets left$  
%		\State $j\gets middle$  
%		\State $k\gets 0$  
%		\State $result \gets 0$  
%		\While{$i<middle$ \textbf{and} $j<right$}  
%		\If{$Array[i]<Array[j]$}  
%		\State $B[k++]\gets Array[i++]$  
%		\Else  
%		\State $B[k++] \gets Array[j++]$  
%		\State $result \gets result + (middle - i)$  
%		\EndIf  
%		\EndWhile  
%		\While{$i<middle$}  
%		\State $B[k++] \gets Array[i++]$  
%		\EndWhile  
%		\While{$j<right$}  
%		\State $B[k++] \gets Array[j++]$  
%		\EndWhile  
%		\For{$i = 0 \to k-1$}  
%		\State $Array[left + i] \gets B[i]$  
%		\EndFor  
%		\State \Return{$result$}  
%		\EndFunction  
%	\end{algorithmic}  
%\end{algorithm}