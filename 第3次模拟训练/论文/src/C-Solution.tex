\mcmSection{模型建立与求解}

% =======================================
% 问题一
% =======================================

\mcmSubsection{问题一:数据收集、数据预处理与描述性分析}

在问题一中,要求在收集相关数据的情况下对国家货币供给量、国家总收入、利率及通货膨胀等数据进行描述性分析。我们通过互联网检索手段获取数据,然后通过Python软件对数据进行预处理,对上述所讨论指标进行绘图分析、统计量的计算,相关具体结果汇入表格,具体操作如下。
	
\mcmSubsubsection{数据收集}

通过互联网检索手段,我们发现国家说数据统计局、世界银行集团及TradingEconomics三个组织拥有相关数据。但因各个机构数据起始年限不一,数据整理格式、精确性不一,考虑到WorldBank提供的数据样本具有良好的格式,同时对比三个数据库的相关数据,发现三者在同一指标下数据波动不大,即使用WordBank提供的数据具有一定的可靠性。

其中世界银行数据库提供的相关操作界面如下:

%\begin{center}
%\begin{figure}[h]
%	\centering
%	\begin{minipage}[t]{0.6\textwidth}
%		\centering
%		\dashbox{\includegraphics[width=\textwidth]{../res/世界银行集团.png}} % 使用虚线框住图片
%		\caption{世界银行单个指标查询结果示意图}
%		\label{fig:WorldBank}
%	\end{minipage}%
%	\hspace*{1cm} % 插入水平空间
%	\begin{minipage}[t]{0.34\textwidth}
%		\raggedright % 左对齐
%		\vspace{-5.4cm} % 调整此值以实现更好的对齐文字顶端和图形顶端
%		\qquad 可以看到,页面右下角有导出CSV文件的导出链接,以及 DataBank 元数据库,我们可以通过点击链接下载或查询相关数据。如果获取的数据量过大,则可通过右键点击检查网页源码定位所需元素,通过 Python \\爬虫技术实现自动化爬取。
%	\end{minipage}
%\end{figure}
%\end{center}
	
\mcmSubsubsection{数据预处理}	
为了对货币供给量、收入、利率与通货膨胀等数据进行描述性分析。我们首先对收集数据进行提取,整合,筛选,补偿,归一化处理。

下列文件列表中,与我们所需数据直接相关的有:M2(广义货币)csv文件,GNI(国民总收入)csv文件,实际利率csv文件以及IR(以CPI为指数计算的通货膨胀率)csv文件。

其在文件列表中的具体名称如图所示,

%\begin{figure}[h]
%	\centering
%	\begin{minipage}[t]{0.6\textwidth}
%		\centering
%		\dashbox{\includegraphics[width=\textwidth]{../res/CSV数据收集列表.png}} % 使用虚线框住图片
%		\caption{各指标下载后文件列表}
%		\label{fig:IndicatorList}
%	\end{minipage}%
%	\hspace*{1cm} % 插入水平空间
%	\begin{minipage}[t]{0.34\textwidth}
%		\raggedright % 左对齐
%		\vspace{-4.7cm} % 调整此值以实现更好的对齐文字顶端和图形顶端
%		\qquad 拓展名以csv结尾的文件是我们的所需文件。而以py结尾的文件是辅助文件,负责列出当前目录下的文件名。所有原始数据文件都放置在代码支撑材料的res文件夹下的csv文件下。
%	\end{minipage}
%\end{figure}

\mcmSubsubsection{数据描述性分析}

利用Python获取上述文件列表中每个csv文件的第46行数据(中国相关的数据),重新组织,形成一个$4 \times 30$的矩阵结构,其相应统计量计算表格如下。

\begin{table}[htbp]
	\centering
	\caption{统计量数据}
	\footnotesize % 对于大表格设置字体为脚注大小或脚本大小或小字体可能效果更好 \scriptsize \small
	\begin{tabular}{@{}lrrrr@{}}
		\toprule
		统计量名称 & M2(单位:元) & GNI(单位:元) & IR(单位:百分号) & RIR(单位:百分号) \\ \midrule
		数据个数 & 30 & 30 & 30 & 30 \\
		平均值 & $8.9239 \times 10^{13}$ & $5.3023 \times 10^{13}$ & $3.34$ & $2.24$ \\
		标准差 & $8.6307 \times 10^{13}$ & $3.4811 \times 10^{13}$ & $5.21$ & $3.29$ \\
		最小值 & $4.6920 \times 10^{12}$ & $1.1446 \times 10^{13}$ & $-1.40$ & $-7.99$ \\
		四分之一分位数 & $1.6155 \times 10^{13}$ & $2.1286 \times 10^{13}$ & $1.02$ & $-0.12$ \\
		四分之二分位数 & $5.4270 \times 10^{13}$ & $4.5314 \times 10^{13}$ & $1.99$ & $2.96$ \\
		四分之三分位数 & $1.5106 \times 10^{14}$ & $8.0646 \times 10^{13}$ & $3.11$ & $4.22$ \\
		最大值 & $2.8734 \times 10^{14}$ & $1.1909 \times 10^{14}$ & $24.26$ & $7.36$ \\
		偏度 & $8.7368 \times 10^{-1}$ & $4.8477 \times 10^{-1}$ & $2.95$ & $-0.95$ \\
		峰度 & $-4.4214 \times 10^{-1}$ & $-1.1427 \times 10^{-1}$ & $9.70$ & $1.79$ \\ 
		\bottomrule
	\end{tabular}
	\label{tab:statistics}
\end{table}

调用绘图命令绘图如下。

%\begin{figure}[H]
%	\centering
%	\begin{minipage}[t]{0.48\linewidth}
%		\subfloat[Bar Plot]{%
%			\includegraphics[width=\linewidth]{../res/1_bar_plot.png}%
%			\label{fig:bar_plot}%
%		}
%		
%		\vspace{0.5em}
%		
%		\subfloat[Box Plot]{%
%			\includegraphics[width=\linewidth]{../res/2_box_plot.png}%
%			\label{fig:box_plot}%
%		}
%		
%		\vspace{0.5em}
%		
%		\subfloat[Area Plot]{%
%			\includegraphics[width=\linewidth]{../res/3_area_plot.png}%
%			\label{fig:area_plot}%
%		}
%	\end{minipage}%
%	\hfill
%	\begin{minipage}[t]{0.48\linewidth}
%		\vspace{1em}
%		\textbf{图 (a) 描述:}\\
%		子图(a)绘出了1994年以后30年的M2,GNI,RIR(实际利率,Real Intrest Rate),IR(以CPI为指数的通货膨胀率, Inflation Rate - CPI)的条形统计图。由图可得,M2和GNI指标逐年递增,而RIR和IR有正有负,与当年的经济状况,国际形势以及意外事故有关。
%		
%		\vspace{4em}
%		
%		\textbf{图 (b) 描述:}\\
%		通过对子图(b)的观察我们可以看到各个指标$\frac{1}{4}$,$\frac{2}{4}$,$\frac{3}{4}$,$\frac{4}{4}$的分位情况,其中,M2及GNI数据分布空间较大,而IR与RIR分布区间较小。需要注意的是,尽管各个指标的量纲不同,但是通过箱线图反映各个指标自身数据四分之比例区间也是具有可比性的。
%		
%		\vspace{4em}
%		
%		\textbf{图 (c) 描述:}\\
%		子图(c)既显示了各个数据随年份的变化趋势也显示了各个指标累计加和的变化情况,各指标的变化趋势实质上与条形图反映一致,不再赘述。而加和情况反映了多年来各个指标的总效益,由图易知,各个指标在时间上纵向求和均为正值,即各个指标在时间维度上加和具有单增趋势。
%	\end{minipage}
%	
%	\caption{不同类型的图及其描述}
%	\label{fig:all_plots}
%\end{figure}

% =======================================
% 问题二
% =======================================

\mcmSubsection{问题二:单变量回归模型预测通货膨胀率与综合多元回归模型预测通货膨胀率}

在问题二中,在已收集数据的情况下,要求我们分别建立对国家货币供给量与通货膨胀率、国家总收入与通货膨胀率、利率与通货膨胀率的数学模型,并利用所建模型预测通货膨胀率的未来走势。

\mcmSubsubsection{广义货币供应量(M2)与通货膨胀率(IR)之间的关系}

通过货币数量方程式MV=PY来建立模型,其中 ( M ) 代表货币供应量,( V ) 是货币流通速度,( P ) 是物价水平,( Y ) 是国内生产总值。假设货币流通速度和产出水平被认为是相对稳定的,因此货币供应量的变化主要通过影响物价水平来体现。

\begin{equation}
	\pi_t = \beta_0 + \beta_1 \Delta M_t + \epsilon_t
\end{equation}

$( \pi_t )$ 是时间 ( t ) 的通货膨胀率。
$( \Delta M_t )$是时间 ( t ) 的货币供应量的变化率。
$( \beta_0 )$是常数。
$( \beta_1 )$ 是货币供应量变化率对通货膨胀率的影响系数。
$( \epsilon_t )$ 是随机误差项。

\mcmSubsubsection{国民总收入(GNI)与通货膨胀率(IR)之间的关系}
\begin{equation}
	Y_t = Y_{t-1} + \alpha \pi_t + \beta X_t + \epsilon_t 
\end{equation}

$( Y_t )$是在时间 ( t ) 的收入水平。
$( Y_{t-1} )$ 是在时间 ( t-1 ) 的收入水平。
$( \pi_t )$ 是时间 ( t ) 的通货膨胀率。
$( \alpha )$ 是通货膨胀率对收入增长的影响系数。
$( X_t )$ 是其他可能影响收入的控制变量。
$( \beta )$ 是控制变量的系数。
$( \epsilon_t )$ 是随机误差项。

\mcmSubsubsection{实际利率(RIR)与通货膨胀率(IR)之间的关系}

名义利率等于实际利率加上通货膨胀率

\begin{equation}
	i_t = \alpha_0 + \alpha_1 \pi_t^e + \epsilon_t
\end{equation}

$(i_t )$是时间 ( t ) 的名义利率。
$( \pi_t^e )$是时间 ( t ) 的通货膨胀预期率。
$( \alpha_0 )$是模型的截距项。
$( \alpha_1 )$是通货膨胀预期率对名义利率的影响系数。
$( \epsilon_t )$是随机误差项。

\mcmSubsubsection{综合模型:多元线性回归表达式}

基于前述三个关系,我们可以构建一个综合的多元线性回归模型,以通货膨胀率 $\pi_t$ 为因变量:

\begin{equation}
\pi_t = \gamma_0 + \gamma_1 (Y_t - Y_{t-1}) + \gamma_2 X_t + \gamma_3 \Delta M_t + \gamma_4 i_t + \gamma_5 \pi_t^e + \epsilon_t
\end{equation}

其中,
$\pi_t$ 是时间 $t$ 的通货膨胀率
$Y_t - Y_{t-1}$ 是收入的变化
$X_t$ 是其他影响收入的控制变量
$\Delta M_t$ 是货币供应量的变化率
$i_t$ 是名义利率
$\pi_t^e$ 是通货膨胀预期率
$\gamma_0, \gamma_1, \gamma_2, \gamma_3, \gamma_4, \gamma_5$ 是待估计的系数
$\epsilon_t$ 是随机误差项。

%\begin{figure}[h]
%	\centering
%	\begin{minipage}[t]{0.6\textwidth}
%		\centering
%		\dashbox{\includegraphics[width=\textwidth]{../res/flowchart.png}} % 使用虚线框住图片
%		\caption{集成学习示意图}
%		\label{fig:flowchart}
%	\end{minipage}%
%	\hspace*{0.5cm} % 插入水平空间
%	\begin{minipage}[t]{0.35\textwidth}
%		\raggedright % 左对齐
%		\vspace{-7cm} % 调整此值以实现更好的对齐
%		左侧是多元线性回归综合模型中使用集成学习方法训练模型示意图。该机器学习以线性回归模型,决策树模型,随机森林模型,梯度提升模型以及支持向量机模型作为基学习器。流程图反映了使用输入数据进行训练模型,然后调用模型对新样本进行预测打分投票,经软投票输出最终预测结果的基本流程。
%		\par % 确保段落结束
%		% 如果需要添加更多文本,可以继续添加
%	\end{minipage}
%\end{figure}

% =======================================
% 问题三
% =======================================

\mcmSubsection{问题三}

在问题三中,题目要求建立M2,GNI,IR,RIR的平稳性检验模型、协整性检验模型及因果关系检验模型,其中平稳性模型我们采取ADF假设检验对每个指标自身随时间的变化进行检验,而协整性和因果性涉及到两个指标间的关系,我们需要两两依次判断。

\mcmSubsubsection{平稳性检验}

对于每个变量(M2、GNI、IR和RIR),我们采用ADF检验。以M2为例,其数学模型如下:

\begin{equation}
	\Delta M2_t = \alpha + \beta t + \gamma M2_{t-1} + \sum_{i=1}^{p} \delta_i \Delta M2_{t-i} + \varepsilon_t
\end{equation}

对GNI、IR和RIR也采用相同形式的方程。

\mcmSubsubsection{协整性检验}

对于协整性检验,我们使用Johansen方法。假设检验M2和GNI之间的协整关系,其VAR模型可表示为:

\begin{equation}
	\Delta \begin{pmatrix} M2_t \\ GNI_t \end{pmatrix} = \Pi \begin{pmatrix} M2_{t-1} \\ GNI_{t-1} \end{pmatrix} + \sum_{i=1}^{p-1} \Gamma_i \Delta \begin{pmatrix} M2_{t-i} \\ GNI_{t-i} \end{pmatrix} + BZ_t + \begin{pmatrix} \varepsilon_{1t} \\ \varepsilon_{2t} \end{pmatrix}
\end{equation}

其中,$\Pi$和$\Gamma_i$是系数矩阵,$Z_t$是确定性变量,$\varepsilon_{1t}$和$\varepsilon_{2t}$是误差项。

对其他变量对也采用类似的方程。

\mcmSubsubsection{因果关系检验}

对于Granger因果关系检验,以M2和GNI为例:

检验M2是否是GNI的Granger原因:

\begin{equation}
	GNI_t = \alpha_0 + \sum_{i=1}^{m} \alpha_i GNI_{t-i} + \sum_{j=1}^{n} \beta_j M2_{t-j} + \varepsilon_t
\end{equation}

检验GNI是否是M2的Granger原因:

\begin{equation}
	M2_t = \gamma_0 + \sum_{i=1}^{p} \gamma_i M2_{t-i} + \sum_{j=1}^{q} \delta_j GNI_{t-j} + \eta_t
\end{equation}

对于其他变量对(如M2和IR、M2和RIR、GNI和IR、GNI和RIR、IR和RIR),我们也采用类似的方程对进行检验。

% =======================================
% 问题四
% =======================================

\mcmSection{问题四}






% =======================================
% 模型评价与改
% =======================================

\mcmSection{模型评价与改进}