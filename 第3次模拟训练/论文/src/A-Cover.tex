\thispagestyle{empty}   % 定义起始页的页眉页脚格式为 empty —— 空,也就没有页眉页脚

\begin{center}
    \textbf{\fontsize{20}{1.5}这里是论文标题}

    \textbf{摘 要}
\end{center}


     本文以我国近年来的经济数据为基础,深入探讨了货币供给量、收入、利率与通货膨胀之间的相互作用机制。首先,通过描述性分析,我们揭示了货币供给量、收入、利率和通货膨胀的基本统计特征和变化趋势。接着,我们分别构建了货币供给量、收入、利率和通货膨胀的预测模型,采用自回归移动平均(ARMA)、多元线性回归、向量自回归(VAR)和指数平滑等模型方法,对未来的走势进行了预测。此外,本文通过平稳性检验、协整性检验和因果关系检验,探讨了这些变量之间的内在联系和相互作用。最后,我们构建了一个多维时间序列模型,将货币供给量、收入、利率与通货膨胀纳入同一框架内进行分析,并通过该模型对未来走势进行了预测。研究结果显示,货币供给量、收入、利率与通货膨胀之间存在显著的关联性,且各变量之间具有协整关系和因果关系。本研究不仅为理解这些宏观经济变量之间的动态关系提供了理论依据,而且为政策制定者提供了实证参考,有助于优化宏观经济调控策略,促进经济平稳健康发展。

     本研究对于丰富和完善我国宏观经济模型,提高经济预测准确性,以及为相关政策制定提供科学依据具有重要意义。我们发现,货币供给量的适度调控对稳定收入和通货膨胀具有积极作用,而利率的调整则对货币供给量和收入水平产生重要影响。因此,在制定宏观经济政策时,需要充分考虑这些变量之间的相互关系和影响,以实现经济的长期稳定发展。\newline \newline



% ==================================================
% @brief    论文摘要
% ==================================================

%\newline \newline
% \newline 与 \\效果相同 
% \linebreak 强制换行

\noindent{\textbf{关键词:} 
	\textbf{宏观经济变量}\quad   \textbf{货币供给}\quad \textbf{收入水平}\quad  \textbf{利率} \quad \textbf{通货膨胀}\quad \textbf{因果关系检验}\quad
	\textbf{多维时间序列模型}}
