% 设置页码计数器为 1 (也就是当前页面为第一页)
\setcounter{page}{1}

% ==================================================
% @brief    问题重述
% ==================================================

\mcmSection{问题重述}

\mcmSubsection{问题背景}

在现代经济体系中,货币供给量、居民收入、市场利率和通货膨胀是四大宏观经济指标,它们相互关联,共同影响着经济的运行和社会的福祉。理解这些指标之间的关系对于政府制定有效的宏观经济政策、维护经济稳定和促进经济发展至关重要。

\mcmSubsection{问题重述}

本研究旨在深入分析和理解我国经济运行中货币供给量、收入、利率与通货膨胀之间的关系,以及它们在经济调控过程中的作用机制。

\textbf {问题一},
收集我国近年来的宏观经济数据,例如货币供给量、居民收入水平、市场利率和通货膨胀率等等进行详细的数据进行描述性的分析。

\textbf {问题二},
基于收集到的数据,先进行数据预处理,确保数据质量,再分别构建针货币供给量、收入、利率和通货膨胀的模型。

\textbf {问题三},
在对货币供给量、收入、利率和通货膨胀的关系进行研究之前,首先需要检验这些时间序列数据的平稳性,以确定是否适合进行后续的建模分析。通过协整性检验,探究这些变量之间是否存在长期的稳定关系。本研究将采用适当的统计方法,如Granger因果关系检验,来分析货币供给量、收入、利率与通货膨胀之间是否存在因果关系,以及这些关系的方向性。

\textbf {问题四} 结合上述分析结果,构建一个包含货币供给量、收入、利率和通货膨胀的多维时间序列模型。并运用该模型进行未来动态预测。

本研究旨在为我国宏观经济调控提供科学依据,帮助政策制定者更好地理解和利用货币供给、收入、利率和通货膨胀之间的关系,以实现经济的稳定增长和通货膨胀的有效控制。


% ==================================================
% @brief    问题分析
% ==================================================

\mcmSection{问题分析}

\mcmSubsection{分析}

\textbf {针对问题一},通过收集我国近年来的宏观经济数据,包括货币供给量、居民收入水平、市场利率和通货膨胀率进行详细的数据等等,包括趋势分析,进行统计量计算以及通过图形工具和相关系数分别进行分布分析和相关性的检验。


\textbf {针对问题二},基于收集到的数据,先进行数据预处理,确保数据质量,再选择合适的模型如VAR或者ARMA等,分别构建针对货币供给量、收入、利率和通货膨胀的模型。这些模型应当能够反映出各自变量的未来走势,并为经济决策提供参考。需要注意经济变量之间的关系可能会随着时间和经济的条件变化而变化。

\textbf {针对问题三},在对货币供给量、收入、利率和通货膨胀的关系进行研究之前,首先需要检验这些时间序列数据的平稳性,以确定是否适合进行后续的建模分析。通常同时,通过协整性检验,探究这些变量之间是否存在长期的稳定关系。本研究将采用适当的统计方法,如Granger因果关系检验,来分析货币供给量、收入、利率与通货膨胀之间是否存在因果关系,以及这些关系的方向性。

\textbf {针对问题四},
结合上述分析结果,进行多维时间序列模型构建: 构建一个包含货币供给量、收入、利率和通货膨胀的多维时间序列模型。该模型应当能够全面反映这些变量之间的动态互动关系,并用于预测未来的经济走势







% ==================================================
% @brief    模型假设
% ==================================================

\mcmSection{模型假设}

\begin{enumerate}
    \item 货币供给量对通货膨胀有直接影响:货币供给量的增加将导致通货膨胀率的上升,货币供给量的减少将导致通货膨胀率的下降。
    \item 收入水平与通货膨胀之间存在关联:居民收入水平的增长将增加消费需求,从而可能推动通货膨胀。收入分配的不平等可能导致通货膨胀压力的增加。
    
    \item 利率对货币供给量和通货膨胀有调节作用:提高利率将减少货币供给量,进而抑制通货膨胀。降低利率将增加货币供给量,可能导致通货膨胀率上升。
    
    \item 变量之间存在长期的均衡关系:货币供给量、收入、利率与通货膨胀之间存在长期的均衡关系,即它们之间的相互作用在长期内会达到一种稳定状态。
    \item 变量之间存在因果关系:货币供给量的变化是通货膨胀变化的原因。收入水平的变化是影响利率决策的重要因素。通货膨胀率的变化可能影响居民的储蓄和投资行为,进而影响收入水平。
    
    \item 经济政策和外部冲击对变量关系有影响:货币政策和财政政策的变化将直接影响货币供给量和利率。外部经济冲击(如全球金融危机、自然灾害等)可能会改变货币供给量、收入、利率与通货膨胀之间的关系。
    \item 变量的动态互动具有时变性:随着时间的推移,货币供给量、收入、利率与通货膨胀之间的动态关系可能会发生变化。
    
 
    
\end{enumerate}

% ==================================================
% @brief    符号说明
% ==================================================

\mcmSection{符号说明及名称定义}

\begin{table}[H] %[h]表示在此处添加浮动体,默认为tbf,即页面顶部、底部和空白处添加
		\captionsetup{skip=4pt} % 设置标题与表格的间距为4pt
		\centering
		\setlength{\arrayrulewidth}{2pt} % 设置表格线条宽度为1pt
		\begin{tabular}{cc} %c表示居中,l表示左对齐,r表示右对齐,中间添加“|”表示竖线
			\hline
			\makebox[0.15\textwidth][c]{符号} & \makebox[0.6\textwidth][c]{说明}  \\ 
			\hline
			
			$\text{变量名}$ & \text {变量含义}  \\
			$\text{变量名}$ & \text {变量含义}  \\	
			$\text{变量名}$ & \text {变量含义}  \\
			
			\hline
		\end{tabular}
		% \hline是横线,采用\makebox设置列宽
	\end{table}
